% This is a supplement to:

% Essential LaTeX - Jon Warbrick 02/88
%                 - W T Hewitt, MCC CGU August 1989

% giving a brief overview of mathematical typesetting using LaTeX
%                      D. P. Carlisle 22/9/89

% modified by Richard Kaye for LaTeX2e

% I reproduce here the copyright notice from Essential LaTeX, the same
% conditions apply to this document.

%   Permission is granted to reproduce the document in any way providing
%   that it is distributed for free, except for any reasonable charges for
%   printing, distribution, staff time, etc.  Direct commercial
%   exploitation is not permitted.  Extracts may be made from this
%   document providing an acknowledgement of the original source is
%   maintained.


\documentclass[a4paper]{article}
\usepackage{latexsym}
\usepackage{amssymb}
\usepackage{mathrsfs}
% 
\newcounter{savesection}
\newcounter{savesubsection}

% commands to do 'LaTeX Manual-like' examples

\newlength{\egwidth}\setlength{\egwidth}{0.45\textwidth}

\newenvironment{eg}%
{\begin{list}{}{\setlength{\leftmargin}{0pt}%
\setlength{\rightmargin}{\leftmargin}}\item[]\footnotesize}%
{\end{list}}

% AmsTeX logo
\newcommand{\AmsTeX}{{$\cal A$}\kern-.1667em\lower.5ex\hbox
{$\cal M$}\kern-.125em{$\cal S$}-\TeX}
\newcommand{\AmsLaTeX}{{$\cal A$}\kern-.1667em\lower.5ex\hbox
{$\cal M$}\kern-.125em{$\cal S$}-\LaTeX}
%
\newenvironment{egbox}%
{\begin{minipage}[t]{\egwidth}}%
{\end{minipage}}

\newcommand{\egstart}{\begin{eg}\begin{egbox}}
\newcommand{\egmid}{\end{egbox}\hfill\begin{egbox}}
\newcommand{\egend}{\end{egbox}\end{eg}}

% one or two other commands

\newcommand{\fn}[1]{\hbox{\tt #1}}
\newcommand{\llo}[1]{(see line #1)}
\newcommand{\lls}[1]{(see lines #1)}
\newcommand{\bs}{$\backslash$}

\title{Essential Mathematical \LaTeXe}
\author{D. P. Carlisle \and Richard Kaye}
\date{12th August 1998}

\begin{document}

\maketitle

\section{Introduction}
This document is a supplement to ``Essential \LaTeXe'', describing
basic mathematical typesetting in \LaTeXe. It makes no attempt at 
completeness so if in doubt {\bf read the manual}.

\section{Math, Display-math and Equation}
Mathematics is treated by \LaTeX\  completely differently from ordinary text.
There are two special {\em modes\/} for mathematics, {\em math mode\/}
and {\em display math mode}.

Math mode commands are surrounded by \verb|\(| and \verb|\)|.
\egstart
Some mathematics set inline \( 2\times 3 =   6 \).
Note that spaces in the input file are ignored in math mode.
\egmid
\begin{verbatim}
... set inline \( 2\times 3 =   6 \).
Note that spaces ...
\end{verbatim}
\egend

Display math mode commands are surrounded by \verb|\[| and \verb|\]|.
\egstart
A larger equation to be displayed on a line by itself.
\[ f(x) = \sum_{i=0}^{\infty}\frac{f^{(i)}(x)}{i!} \]
\egmid
\begin{verbatim}
... to be displayed on a line by itself.
\[ f(x) = 
 \sum_{i=0}^{\infty}\frac{f^{(i)}(x)}{i!} \]
\end{verbatim}
\egend

There is a variant of Display math mode, the equation environment
which automatically generates an equation number.

\egstart
\begin{equation}
\left(\begin{array}{cc}
1 & 2 \\0 & 1
\end{array}\right)
\left(\begin{array}{cc}
2 & 0\\1 & 3
\end{array}\right)
=
\left(\begin{array}{cc}
4 & 6 \\1 & 3
\end{array}\right)
\end{equation}
\egmid
\begin{verbatim}
\begin{equation}
\left(\begin{array}{cc}
1 & 2 \\0 & 1
\end{array}\right)
\left(\begin{array}{cc}
2 & 0\\1 & 3
\end{array}\right)
=
\left(\begin{array}{cc}
4 & 6 \\1 & 3
\end{array}\right)
\end{equation}
\end{verbatim}
\egend

The short examples above show the main types of commands available 
in math mode.
\begin{enumerate}
\item Subscripts and superscripts are produced with \verb|_| and \verb|^|, as in
      \verb|x_{1} = p^{2}| \(x_{1} = p^{2}\). 
\item Fractions are produced by the \verb|\frac| command. 
      \verb|\(\frac{a + b}{c}\)| \(\frac{a + b}{c}\).
\item Various commands give names to mathematical symbols.\\
      \verb| \infty \Rightarrow \surd \bigotimes | 
      \(\infty\;\Rightarrow\;\surd\;\bigotimes\)
\item Arrays are produced by the array environment. This is identical
      to the tabular environment described in {\em essential \LaTeX\/}
      except that the entries are typeset in math mode instead of LR mode.
      Note that the array environment does not put brackets arround the array
      so it can also be used for setting determinants, or even 
      sets of equations in which you want the columns to line up.
\item The commands \verb|\left| and \verb|\right| produce delimiters which
      grow as large as needed, they can be used with a variety of symbols,
      e.g., \verb|\left(|, \verb|\left\{|, \verb!\left|!. The full set of 
      these delimiters is shown in Table \ref{dels} below.  There are
      some things to note: 
      \begin{enumerate}
      \item \verb|\left| and \verb|\right| must come in matching
      pairs, otherwise \LaTeX\ won't know what formula the delimiter
      must be adjusted to fit round; 
      \item you must remember to put a delimiter after \verb|\left| 
      and \verb|\right|; 
      \item the delimiters themselves needn't be matched, so 
      \verb|\( \left( ... \right] \)|
      is OK; 
      \item if you \textit{really} don't want any delimiter on
      one side, use `.' in place of a delimiter, as in 
      \verb|\( \left\{ ... \right. \)| (which is often used for definitions
      by cases).
      \end{enumerate}
\end{enumerate}

\section{Spacing}
All spaces in the input file are ignored in math mode. Sometimes
you may want to adjust the spacing. Use one of the following commands
or an explicit \verb|\hspace|.
\begin{center}
\begin{tabular}{ll}
\verb|\,| thin space          &\verb|\:| medium space \\
\verb|\!| negative thin space &\verb|\;| thick space \\
\end{tabular}
\end{center}

A good example of where \LaTeX\ needs some help with spacing is
\egstart
$\int\!\!\int z\, dx dy$ \ instead of $\int\int z dx dy$
\egmid
\verb|\int\!\!\int z\, dx dy .. \int\int z dx dy|.
\egend

\section{Changing fonts in math mode}
The default math mode font is $math\ italic$. This should not be
confused with ordinary {\em text italic}.  The letters are
a slightly different shape, and the spacing between them is 
quite different.
The font for `ordinary' letters can be changed with the 
commands, \verb|\mathbf|, \verb|\mathrm|, \verb|\mathit|,
\verb|\mathsf|, and \verb|\mathnormal|.
These commands take a single argument---the letters that
should be bold or roman or whatever.
Note that lower case Greek letters
are just regarded as mathematical symbols, \verb|\alpha| etc, and are
not affected by these commands.

\verb|\bf| produces {\bf bold face roman} letters. If you wish
to have {\boldmath $bold$ $face$ $math$ $italic$} letters, and bold face 
Greek letters and mathematical symbols, use the \verb|\boldmath| command
{\em before\/} going into math mode. 
This changes the default math fonts to bold.

\egstart{\normalsize
\( x = 2\pi \Rightarrow x \simeq 6.28 \)\\
\(\mathbf{ x = 2\pi \Rightarrow x \simeq 6.28 } \)\\\mbox{}\\
\boldmath
\( x = 2\pi \Rightarrow x \simeq 6.28 \)}
\egmid
\begin{verbatim}
\( x = 2\pi \Rightarrow x \simeq 6.28 \)
\(\mathbf{ x = 2\pi \Rightarrow x \simeq 6.28 } \)
\boldmath
\( x = 2\pi \Rightarrow x \simeq 6.28 \)
\end{verbatim}
\egend

There is also a calligraphic font for upper case letters
produced by the \verb|\mathcal| command.

\egstart
\( \mathcal{F}\mathcal{G} \)
\egmid
\verb|\( \mathcal{F}\mathcal{G} \)|
\egend

\AmsLaTeX\ provides more commands for bold-face mathematics, so that
bold math italic and ordinary math fonts can be mixed in the same
formula without using \verb|\boldmath|, and it also adds `blackboard bold'
characters (\verb|\mathbb|), $\mathbb{NZQRC}$, 
`script', and `fraktur' fonts (\verb|\mathfrak|). $\mathfrak{ABCDEFG}$.  
(RK adds: I'm not very fond of the AMS's script
font though.  I much prefer the oldfashioned copper-plate letters
which are available in the `mathrsfs' package, and which gives---using the
\verb|\mathscr| command---$\mathscr{ABCDEFG}$.)

\section{Symbols}
The tables show most of the symbols available from the standard \LaTeX\ symbol 
fonts. Negations of the relational symbols can be made with the \verb|\not| 
command.

\egstart
$ G \not\equiv H $ 
\egmid
\verb|G \not\equiv H|
\egend

A very common mistake is to ignore the difference between
the various `kinds' of symbols.  If you check the tables, you'll
notice that some symbols are available in different ways.  The 
difference is the spacing on either side of the symbol.
For example to get a colon in math mode, you type \verb|:| or
\verb|\colon|.  The second is a puctuation symbol, the first is 
a relation symbol.  Here are examples of their \textit{proper} use.
\egstart
\( f\colon A\to B \)\\
\( \{ x : x \not\in x \} \)
\egmid
\verb|\( f\colon A\to B \)|
\verb|\( \{ x : x \not\in x \} \)|
\egend

The standard \LaTeX\ fonts have enough symbols for most people, see
Tables~1--\ref{other}.
If you need more you could try the \AmsTeX\ extra symbol fonts, most easily
obtained by loading \verb|amssymb| using the \verb|\usepackage| command.
See Tables \ref{ams}~to \ref{bbams}.  Much more control over mathematical
typesetting is available as a series of \AmsLaTeX{} packages you can
also load.

\section{What about \$'s?}
If you have converted to \LaTeX\ from plain or \AmsTeX, you will
probably be wondering why there has been no mention of \verb|$| 
and \verb|$$|.

In these systems math mode is surrounded by \verb|$|'s and display math mode
is surrounded by \verb|$$|. This has certain drawbacks over the \LaTeX\ system
as it is difficult for your text editor to match \verb|$|'s as it is hard to 
tell which ones are starting math mode and which are ending it. \TeX\ will also
get confused if you miss a \$ out.

The (incorrect) input 
\begin{verbatim}
let (a,b,c)$ be a Pythagorean triple, i.e.\ three 
integers such that $a^{2}+b^{2}=c^{2}$.
\end{verbatim}
produces the slightly mysterious error message
\begin{verbatim}
! Missing $ inserted.
<inserted text> 
		$
<to be read again> 
		   ^
l.56 ...triple, i.e.\ three integers such that $a^
						  {2}+b^{2}=c^{2}$
? 
\end{verbatim}

Note that it reports the wrong error and in the wrong place, the use of the
\verb|^| command out of math mode. \TeX\ has typeset `be a \ldots\ such that'
in math mode and {\em exited\/} math mode at the \verb|$| after `such that'. 
If you had made the equivalent \LaTeX\ error, \LaTeX\ 
has a better idea of what you indended:
\begin{verbatim}
let (a,b,c)\) be a Pythagorean triple, i.e.\ three 
integers such that \(a^{2}+b^{2}=c^{2}\)
\end{verbatim}
The error message may still be unintelligable, but at least
it reports the error in the right place, you have used \verb|\)|
to end math mode when you were not in math mode (as you omitted the 
\verb|\(| which should have been before the {\tt (a,b,c)}).
\begin{verbatim}
LaTeX error.  See LaTeX manual for explanation.
	      Type  H <return>  for immediate help.
! Bad math environment delimiter.
\@latexerr ...for immediate help.}\errmessage {#1}
						  
\)...ifinner $\else \@badmath \fi \else \@badmath 
						  \fi 
l.56 let (a,b,c)\)
		   be a Pythagorean triple, i.e.\ three integers such that \...

? 
\end{verbatim}

The single dollar is sometimes useful for small
sections of math.
\egstart
Let $G$ be a $p$-group
\egmid
\verb|Let $G$ be a $p$-group|
\egend

The double dollar is {\em not\/} always equivalent to \verb|\[ ... \]|,
and so should not be used if you want your \LaTeX\ file to be compatible
with different styles and style options (try the fleqn style option).

\clearpage

\def\W#1#2{$#1{#2}$ &\tt\string#1\string{#2\string}}
\def\X#1{$#1$ &\tt\string#1}
\def\Y#1{$\big#1$ &\tt\string#1}
\def\Z#1{\tt\string#1}


\begin{table}[p]
\centering
\begin{tabular}{*8l}
\X\alpha        &\X\theta       &\X o           &\X\tau         \\
\X\beta         &\X\vartheta    &\X\pi          &\X\upsilon     \\
\X\gamma        &\X\gamma       &\X\varpi       &\X\phi         \\
\X\delta        &\X\kappa       &\X\rho         &\X\varphi      \\
\X\epsilon      &\X\lambda      &\X\varrho      &\X\chi         \\
\X\varepsilon   &\X\mu          &\X\sigma       &\X\psi         \\
\X\zeta         &\X\nu          &\X\varsigma    &\X\omega       \\
\X\eta          &\X\xi                                          \\
								\\
\X\Gamma        &\X\Lambda      &\X\Sigma       &\X\Psi         \\
\X\Delta        &\X\Xi          &\X\Upsilon     &\X\Omega       \\
\X\Theta        &\X\Pi          &\X\Phi
\end{tabular}
\caption{Greek Letters}\label{greek}
\end{table}

\begin{table}
\centering
\begin{tabular}{*8l}
\X\pm           &\X\cap         &\X\diamond             &\X\oplus       \\
\X\mp           &\X\cup         &\X\bigtriangleup       &\X\ominus      \\
\X\times        &\X\uplus       &\X\bigtriangledown     &\X\otimes      \\
\X\div          &\X\sqcap       &\X\triangleleft        &\X\oslash      \\
\X\ast          &\X\sqcup       &\X\triangleright       &\X\odot        \\
\X\star         &\X\vee         &\X\bigcirc     &\X\amalg \\
\X\circ         &\X\wedge       &\X\dagger      &\X\wr      \\
\X\bullet       &\X\setminus    &\X\ddagger     &\X\cdot    \\
\X+             &\X-
\end{tabular}
\caption{Binary Operation Symbols}\label{bin}
\end{table}


\begin{table}
\centering
\begin{tabular}{*8l}
\X\leq          &\X\geq         &\X\equiv       &\X\models      \\
\X\prec         &\X\succ        &\X\sim         &\X\perp        \\
\X\preceq       &\X\succeq      &\X\simeq       &\X\mid         \\
\X\ll           &\X\gg          &\X\asymp       &\X\parallel    \\
\X\subset       &\X\supset      &\X\approx      &\X\bowtie      \\
\X\subseteq     &\X\supseteq    &\X\cong        &\X\propto      \\
\X\neq          &\X\smile       &\X\vdash       &\X\dashv       \\
\X\sqsubseteq   &\X\sqsupseteq  &\X\doteq       &\X\frown       \\
\X\in           &\X\ni          &\X=            &\X>            \\
\X<             &$:$&:
\end{tabular}
\caption{Relation Symbols}\label{rel}
\end{table}

\begin{table}
\centering
\begin{tabular}{*{5}{lp{3.2em}}}
\X,     &\X;    &\X\colon       &\X\ldotp       &\X\cdotp
\end{tabular}
\caption{Punctuation Symbols}\label{punct}
\end{table}

\begin{table}
\centering
\begin{tabular}{*6l}
\X\leftarrow            &\X\longleftarrow       &\X\uparrow     \\
\X\Leftarrow            &\X\Longleftarrow       &\X\Uparrow     \\
\X\rightarrow           &\X\longrightarrow      &\X\downarrow   \\
\X\Rightarrow           &\X\Longrightarrow      &\X\Downarrow   \\
\X\leftrightarrow       &\X\longleftrightarrow  &\X\updownarrow \\
\X\Leftrightarrow       &\X\Longleftrightarrow  &\X\Updownarrow \\
\X\mapsto               &\X\longmapsto          &\X\nearrow     \\
\X\hookleftarrow        &\X\hookrightarrow      &\X\searrow     \\
\X\leftharpoonup        &\X\rightharpoonup      &\X\swarrow     \\
\X\leftharpoondown      &\X\rightharpoondown    &\X\nwarrow
\end{tabular}
\caption{Arrow Symbols}
\end{table}

\begin{table}
\centering
\begin{tabular}{*8l}
\X\ldots        &\X\cdots       &\X\vdots       &\X\ddots       \\
\X\aleph        &\X\prime       &\X\forall      &\X\infty       \\
\X\hbar         &\X\emptyset    &\X\exists      &\X\spadesuit   \\
\X\imath        &\X\nabla       &\X\neg         &\X\heartsuit   \\
\X\jmath        &\X\surd        &\X\flat        &\X\diamondsuit \\
\X\ell          &\X\top         &\X\natural     &\X\clubsuit    \\
\X\wp           &\X\bot         &\X\sharp       &\X\partial     \\
\X\Re           &\X\|           &\X\backslash   &\X\triangle    \\
\X\Im           &\X\angle       &\X.            &\X|
\end{tabular}
\caption{Miscellaneous Symbols}\label{ord}
\end{table}

\begin{table}
\centering
\begin{tabular}{*6l}
\X\sum          &\X\bigcap      &\X\bigodot     \\
\X\prod         &\X\bigcup      &\X\bigotimes   \\
\X\coprod       &\X\bigsqcup    &\X\bigoplus    \\
\X\int          &\X\bigvee      &\X\biguplus    \\
\X\oint         &\X\bigwedge    
\end{tabular}
\caption{Variable-sized  Symbols}\label{op}
\end{table}


\begin{table}
\centering
\begin{tabular}{*8l}
\Z\arccos &\Z\cos  &\Z\csc &\Z\exp &\Z\ker    &\Z\limsup &\Z\min &\Z\sinh \\
\Z\arcsin &\Z\cosh &\Z\deg &\Z\gcd &\Z\lg     &\Z\ln     &\Z\Pr  &\Z\sup  \\
\Z\arctan &\Z\cot  &\Z\det &\Z\hom &\Z\lim    &\Z\log    &\Z\sec &\Z\tan  \\
\Z\arg    &\Z\coth &\Z\dim &\Z\inf &\Z\liminf &\Z\max    &\Z\sin &\Z\tanh
\end{tabular}
\caption{Log-like Symbols}\label{log}
\end{table}


\begin{table}
\centering
\begin{tabular}{*8l}
\X(             &\X)            &\X\uparrow     &\X\Uparrow     \\
\X[             &\X]            &\X\downarrow   &\X\Downarrow   \\
\X\{            &\X\}           &\X\updownarrow &\X\Updownarrow \\
\X\lfloor       &\X\rfloor      &\X\lceil       &\X\rceil       \\
\X\langle       &\X\rangle      &\X/            &\X\backslash   \\
\X|             &\X\|
\end{tabular}
\caption{Delimiters\label{dels}}
\end{table}

\begin{table}
\centering
\begin{tabular}{*8l}
\Y\rmoustache&  \Y\lmoustache&  \Y\rgroup&      \Y\lgroup\\[5pt]
\Y\arrowvert&   \Y\Arrowvert&   \Y\bracevert
\end{tabular}
\caption{Large Delimiters\label{ldels}}
\end{table}

\begin{table}
\centering
\begin{tabular}{*{10}l}
\W\hat{a}       &\W\acute{a}    &\W\bar{a}      &\W\dot{a}      &\W\breve{a}    \\
\W\check{a}     &\W\grave{a}    &\W\vec{a}      &\W\ddot{a}     &\W\tilde{a}    \\
\end{tabular}
\caption{Math mode accents}\label{accent}
\end{table}

\begin{table}
\centering
\begin{tabular}{*4l}
\W\widetilde{abc}       &\W\widehat{abc}                \\
\W\overleftarrow{abc}   &\W\overrightarrow{abc}         \\
\W\overline{abc}        &\W\underline{abc}              \\
\W\overbrace{abc}       &\W\underbrace{abc}             \\[5pt]
\W\sqrt{abc}            &$\sqrt[n]{abc}$&\verb|\sqrt[n]{abc}|\\
$f'$&\verb|f'|          &$\frac{abc}{xyz}$&\verb|\frac{abc}{xyz}|
\end{tabular}
\caption{Some other constructions}
\end{table}

\begin{table}
\centering
\begin{tabular}{*4l}
\X\mho             &\X\Join            \\
\X\Box             &\X\Diamond         \\
\X\sqsubset        &\X\sqsupset        \\
\X\lhd             &\X\rhd             \\
\X\unlhd           &\X\unrhd           \\
\X\leadsto
\end{tabular}
\caption{Further symbols available with the \texttt{latexsym} package}\label{other}
\end{table}

\begin{table}
\centering
\begin{tabular}{*4l}
\X\boxdot               &\X\boxplus \\ 
	\X\boxtimes            &\X\square              \\
\X\blacksquare          &\X\centerdot\\ 
	\X\lozenge             &\X\blacklozenge        \\
\X\circlearrowright     &\X\circlearrowleft\\ 
	\X\rightleftharpoons   &\X\leftrightharpoons   \\
\X\boxminus             &\X\Vdash\\ 
	\X\Vvdash              &\X\vDash               \\
\X\twoheadrightarrow    &\X\twoheadleftarrow\\ 
	\X\leftleftarrows      &\X\rightrightarrows    \\
\X\upuparrows           &\X\downdownarrows\\ 
	\X\upharpoonright      &\X\downharpoonright    \\
\X\upharpoonleft        &\X\downharpoonleft\\ 
	\X\rightarrowtail      &\X\leftarrowtail       \\
\X\leftrightarrows      &\X\rightleftarrows\\ 
	\X\Lsh                 &\X\Rsh                 \\
\end{tabular}
\caption{Some AMS symbols\label{ams}}
\end{table}

\begin{table}
\centering
\begin{tabular}{*4l}
\X\rightsquigarrow      &\X\leftrightsquigarrow\\ 
	\X\looparrowleft       &\X\looparrowright      \\
\X\circeq               &\X\succsim\\ 
	\X\gtrsim              &\X\gtrapprox           \\
\X\multimap             &\X\therefore\\ 
	\X\because             &\X\doteqdot            \\
\X\triangleq            &\X\precsim\\ 
	\X\lesssim             &\X\lessapprox          \\
\X\eqslantless          &\X\eqslantgtr\\ 
	\X\curlyeqprec         &\X\curlyeqsucc         \\
\X\preccurlyeq          &\X\leqq\\ 
	\X\leqslant            &\X\lessgtr             \\
\X\backprime            &\X\risingdotseq\\ 
	\X\fallingdotseq       &\X\succcurlyeq         \\
\X\geqq                 &\X\geqslant\\ 
	\X\gtrless             &\X\sqsubset            \\
\X\sqsupset             &\X\vartriangleright\\ 
	\X\vartriangleleft     &\X\trianglerighteq     \\
\X\trianglelefteq       &\X\bigstar\\ 
	\X\between             &\X\blacktriangledown   \\
\X\blacktriangleright   &\X\blacktriangleleft\\ 
	\X\vartriangle         &\X\blacktriangle       \\
\X\triangledown         &\X\eqcirc\\ 
	\X\lesseqgtr           &\X\gtreqless           \\
\X\lesseqqgtr           &\X\gtreqqless\\ 
	\X\Rrightarrow         &\X\Lleftarrow          \\
\X\veebar               &\X\barwedge\\ 
	\X\doublebarwedge      &\X\angle               \\
\X\measuredangle        &\X\sphericalangle\\ 
	\X\varpropto           &\X\smallsmile          \\
\X\smallfrown           &\X\Subset\\ 
	\X\Supset              &\X\Cup                 \\
\X\Cap                  &\X\curlywedge\\ 
	\X\curlyvee            &\X\leftthreetimes      \\
\X\rightthreetimes      &\X\subseteqq\\ 
	\X\supseteqq           &\X\bumpeq              \\
\X\Bumpeq               &\X\lll\\ 
	\X\ggg                 &\X\circledS            \\
\X\pitchfork            &\X\dotplus\\ 
	\X\backsim             &\X\backsimeq           \\
\X\complement           &\X\intercal\\ 
	\X\circledcirc         &\X\circledast          \\
\X\circleddash          &\X\ulcorner\\ 
	\X\urcorner            &\X\llcorner            \\
\X\lrcorner             &\X\yen\\ 
	\X\checkmark           &\X\circledR            \\
\X\maltese              &\X\lvertneqq\\ 
	\X\gvertneqq           &\X\nleq                \\
\end{tabular}
\caption{More AMS symbols\label{mams}}
\end{table}

\begin{table}
\centering
\begin{tabular}{*4l}

\X\ngeq                 &\X\nless\\ 
	\X\ngtr                &\X\nprec               \\
\X\nsucc                &\X\lneqq\\ 
	\X\gneqq               &\X\nleqslant           \\
\X\ngeqslant            &\X\lneq\\ 
	\X\gneq                &\X\npreceq             \\
\X\nsucceq              &\X\precnsim\\ 
	\X\succnsim            &\X\lnsim               \\
\X\gnsim                &\X\nleqq\\ 
	\X\ngeqq               &\X\precneqq            \\
\X\succneqq             &\X\precnapprox\\ 
	\X\succnapprox         &\X\lnapprox            \\
\X\gnapprox             &\X\nsim\\ 
	\X\ncong               &\X\varsubsetneq        \\
\X\varsupsetneq         &\X\nsubseteqq\\ 
	\X\nsupseteqq          &\X\subsetneqq          \\
\X\supsetneqq           &\X\varsubsetneqq\\ 
	\X\varsupsetneqq       &\X\subsetneq           \\
\X\supsetneq            &\X\nsubseteq\\ 
	\X\nsupseteq           &\X\nparallel           \\
\X\nmid                 &\X\nshortmid\\ 
	\X\nshortparallel      &\X\nvdash              \\
\X\nVdash               &\X\nvDash\\ 
	\X\nVDash              &\X\ntrianglerighteq    \\
\X\ntrianglelefteq      &\X\ntriangleleft\\ 
	\X\ntriangleright      &\X\nleftarrow          \\
\X\nrightarrow          &\X\nLeftarrow\\ 
	\X\nRightarrow         &\X\nLeftrightarrow     \\
\X\nleftrightarrow      &\X\divideontimes\\ 
	\X\varnothing          &\X\nexists             \\
\X\mho                  &\\% was: \X\thorn\\ 
	\X\beth                &\X\gimel               \\
\X\daleth               &\X\lessdot\\ 
	\X\gtrdot              &\X\ltimes              \\
\X\rtimes               &\X\shortmid\\ 
	\X\shortparallel       &\X\smallsetminus       \\
\X\thicksim             &\X\thickapprox\\ 
	\X\approxeq            &\X\succapprox          \\
\X\precapprox           &\X\curvearrowleft\\ 
	\X\curvearrowright     &\X\digamma             \\
\X\varkappa             &\X\hslash\\ 
	\X\hbar                &\X\backepsilon
\end{tabular}
\caption{Even more AMS symbols\label{emams}}
\end{table}

\begin{table}
\centering
\begin{tabular}{*2l}
$\mathbb{A~B~C~D~E~F}$ & \verb|\mathbb A \mathbb B ... |\\
$\mathfrak{A~B~C~D~E~F}$ & \verb|\mathfrak A \mathfrak B ... |\\
$\mathscr{A~B~C~D~E~F}$ &\verb|\mathscr A \mathscr B ... |
\end{tabular}
\caption{Other symbol fonts available with \texttt{amssymb} and \texttt{mathrsfs}}\label{bbams}
\end{table}



\end{document}
