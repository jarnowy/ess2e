% Essential LaTeX 2e - Jon Warbrick 02/88, Richard Kaye 10/95, 5/97

% Modification of `Essential LaTeX' by Jon Warbrick, modification
% made by Richard Kaye to run under LaTeX2e in native mode,
% with some deletions, simplifications and additions.

\documentclass[11pt,a4paper]{article}

\usepackage[margin=1in]{geometry}

% el2emath

% AmsTeX logo
\newcommand{\AmsTeX}{{$\cal A$}\kern-.1667em\lower.5ex\hbox
{$\cal M$}\kern-.125em{$\cal S$}-\TeX}
\newcommand{\AmsLaTeX}{{$\cal A$}\kern-.1667em\lower.5ex\hbox
{$\cal M$}\kern-.125em{$\cal S$}-\LaTeX}
%

\usepackage{latexsym}
\usepackage{amssymb}
\usepackage{mathrsfs}
% 
\usepackage{docmute}

% crossref

\newtheorem{thm}{Theorem}[section]
\newtheorem{chall}[thm]{Challenge}

% extraref

\usepackage{makeidx}
\newcommand{\BibTeX}{{\sc Bib}\TeX}

% el2emore

% \newcommand{\fn}[1]{{\tt #1}}
\newcommand{\cn}[1]{{\tt \char"5C #1}}

% counters used for the sample file example

\newcounter{savesection}
\newcounter{savesubsection}

% commands to do 'LaTeX Manual-like' examples

\newlength{\egwidth}\setlength{\egwidth}{0.42\textwidth}

\newenvironment{eg}%
{\begin{list}{}{\setlength{\leftmargin}{0.05\textwidth}%
\setlength{\rightmargin}{\leftmargin}}\item[]\footnotesize}%
{\end{list}}

\newenvironment{egbox}%
{\begin{minipage}[t]{\egwidth}}%
{\end{minipage}}

\newcommand{\egstart}{\begin{eg}\begin{egbox}}
\newcommand{\egmid}{\end{egbox}\hfill\begin{egbox}}
\newcommand{\egend}{\end{egbox}\end{eg}}

% one or two other commands

\newcommand{\fn}[1]{\hbox{\tt #1}}
\newcommand{\llo}[1]{(see line #1)}
\newcommand{\lls}[1]{(see lines #1)}
\newcommand{\bs}{$\backslash$}

\title{Essential \LaTeXe}
\author{Jon Warbrick, David Carlisle, Richard Kaye}
\date{12th August 1998}


\begin{document}

\maketitle

\tableofcontents

\section{Introduction}

This document is an attempt to give you all the essential
information that you will need in order to use the \LaTeX{} Document
Preparation System.  Only very basic features are covered, and a
vast amount of detail has been omitted.  In a document of this size
it is not possible to include everything that you might need to know,
and if you intend to make extensive use of the program you should
refer to a more complete reference.  Attempting to produce complex
documents using only the information found below will require
much more work than it should, and will probably produce a less
than satisfactory result.

The original reference for \LaTeX{} is {\em The \LaTeX{} User's guide and
Reference Manual\/} by Leslie Lamport.  A subsequent edition
covers \LaTeXe.  There will be a certain amount of information about
using \LaTeX{} available on your particular machine; this is referred
to here as the `local guide'.

\section{How does \LaTeX{} work?}

In order to use \LaTeX{} you generate a file containing
both the text that you wish to print and instructions to tell \LaTeX{}
how you want it to appear.  You will normally create
this file using a text editor or word processor available
on your system.  (Note that if you use a word-processor, you must save
the file in \textit{text} or \textit{ASCII} format.).  
You can give the file any name you
like, but it should end ``\fn{.tex}'' to identify the file's contents.
You then get \LaTeX{} to process the file, and it creates a
new file of typesetting commands; this has the same name as your file but
the ``\fn{.tex}'' ending is replaced by ``\fn{.dvi}''.  This stands for
`\textit{D}e\textit{v}ice \textit{I}ndependent' and, as the name implies, 
this file can be used to create output on a range of printing devices.

Thus, producing a final \LaTeX{} document involves three
processes: editing the original \fn{.tex}~file; processing this
file using \LaTeX, to give a \fn{.dvi} file; and finally viewing or
printing the \fn{.dvi}~file.  In all likelihood, you will have made
mistakes in the original typing, and will have to return to the
editor to correct these, and so the processes are repeated until
you are satisfied with your final output.\footnote{The 
main reason why \LaTeX{} is designed to work 
like this is to ensure that it works on a huge variety of machines.
Text (or ASCII) files can be produced on all machines, and \fn{.dvi}
files are special binary files using a standard format which enables
the printing or viewing programs (so-called \textit{drivers}) to
output the file in a readable form on devices capable of resolutions
from only 75 dots per inch to 2400 dots per inch or even higher.
The portability of \LaTeX{} and its documents is the main reason for its
success and popularity.}

One side-effect of the way \LaTeX{} has been designed is that
when you type your text, you need to type \textit{special commands}
to obtain special effects. For users used to
word-processors where you see the document on the screen as
you type it, this can take some time to get used to, but in the
long run, and especially for technical documents,
the use of \LaTeX's commands will save you time and give you 
better quality output.

Rather than encourage you to dictate exactly how your document
should be laid out, \LaTeX{} instructions allow you describe its
{\em logical structure\/}.  For example, you can think of a quotation
embedded within your text as an element of this logical structure: 
you would normally expect a quotation to be displayed in a recognisable 
style to set it off from the rest of the text.
A human typesetter would recognise the quotation and handle
it accordingly, but since \LaTeX{} is only a computer program it requires
your help.  There are therefore \LaTeX{} commands that allow you to
identify quotations and as a result allow \LaTeX{} to typeset them 
correctly.

Fundamental to \LaTeX{} is the idea of a {\em document style\/} that
determines exactly how a document will be formatted.  \LaTeX{} provides
standard document styles that describe how standard logical structures
(such as quotations) should be formatted.  You may have to supplement
these styles by specifying the formatting of logical structures
peculiar to your document, such as mathematical formulae.  You can
also modify the standard document styles or even create an entirely
new one, though you should know the basic principles of typographical
design before creating a radically new style.

There are a number of good reasons for concentrating on the logical
structure rather than on the appearance of a document.  It prevents
you from making elementary typographical errors in the mistaken
idea that they improve the aesthetics of a document---you should
remember that the primary function of document design is to make
documents easier to read, not prettier.  It is more flexible, since
you only need to alter the definition of the quotation style
to change the appearance of all the quotations in a document.  Most
important of all, logical design encourages better writing.
A visual system makes it easier to create visual effects rather than
a coherent structure; logical design encourages you to concentrate on
your writing and makes it harder to use formatting as a substitute
for good writing.

\section{A Sample \LaTeX{} file}

Have a look at the example \LaTeX{} file in Figure~\ref{fig:sample}.  It
is a slightly modified copy of the standard \LaTeX{} example file
\fn{small2e.tex}.  Your local guide will tell you how you can make
your own copy of this file.  The line numbers down the left-hand side
are not part of the file, but have been added to make it easier to
identify various portions. Also have a look at Figure~\ref{fig:result} which
shows, more or less, the result of processing this file.

\begin{figure} %---------------------------------------------------------------
{\footnotesize\begin{verbatim}
 1: % This is a small sample LaTeX input file (Version of 10 April 1994)
 2: %
 3: % Use this file as a model for making your own LaTeX input file.
 4: % Everything to the right of a  %  is a remark to you and is ignored by LaTeX.
 5: 
 6: % The Local Guide tells how to run LaTeX.
 7: 
 8: % WARNING!  Do not type any of the following 10 characters except as directed:
 9: %                &   $   #   %   _   {   }   ^   ~   \   
10: 
11: \documentclass{article}        % Your input file must contain these two lines 
12: \begin{document}               % plus the \end{document} command at the end.
13: 
14: 
15: \section{Simple Text}          % This command makes a section title.
16: 
17: Words are separated by one or more spaces.  Paragraphs are separated by
18: one or more blank lines.  The output is not affected by adding extra
19: spaces or extra blank lines to the input file.
20: 
21: Double quotes are typed like this: ``quoted text''.
22: Single quotes are typed like this: `single-quoted text'.
23: 
24: Long dashes are typed as three dash characters---like this.
25: 
26: Emphasized text is typed like this: \emph{this is emphasized}.
27: Bold       text is typed like this: \textbf{this is bold}.
28: 
29: \subsection{A Warning or Two}  % This command makes a subsection title.
30: 
31: If you get too much space after a mid-sentence period---abbreviations
32: like etc.\ are the common culprits)---then type a backslash followed by
33: a space after the period, as in this sentence.
34: 
35: Remember, don't type the 10 special characters (such as dollar sign and
36: backslash) except as directed!  The following seven are printed by
37: typing a backslash in front of them:  \$  \&  \#  \%  \_  \{  and  \}.  
38: The manual tells how to make other symbols.
39: 
40: \end{document}                 % The input file ends with this command.
\end{verbatim}  }

\caption{A Sample \LaTeX{} File}\label{fig:sample}

\end{figure} %-----------------------------------------------------------------

\begin{figure} %---------------------------------------------------------------

% This figure will only work if the main style of this document
% is article, or something like article, because that's what the example
% file thinks it is.  Unfortunately, I can't see how to include a
% \documentstyle command in a figure!

\setcounter{savesection}{\value{section}}
\setcounter{section}{0}
\setcounter{savesubsection}{\value{subsection}}
\setcounter{subsection}{0}

\setlength{\parindent}{17pt}
\setlength{\parskip}{0pt}

\noindent\rule{\textwidth}{0.8pt}

% This is a small sample LaTeX input file (Version of 10 April 1994)
%
% Use this file as a model for making your own LaTeX input file.
% Everything to the right of a  %  is a remark to you and is ignored by LaTeX.

% The Local Guide tells how to run LaTeX.

% WARNING!  Do not type any of the following 10 characters except as directed:
%                &   $   #   %   _   {   }   ^   ~   \   

%------------------------commented out!------------------------
%\documentclass{article}        % Your input file must contain these two lines 
%\begin{document}               % plus the \end{document} command at the end.
%------------------------commented out!------------------------


\section{Simple Text}          % This command makes a section title.

Words are separated by one or more spaces.  Paragraphs are separated by
one or more blank lines.  The output is not affected by adding extra
spaces or extra blank lines to the input file.

Double quotes are typed like this: ``quoted text''.
Single quotes are typed like this: `single-quoted text'.

Long dashes are typed as three dash characters---like this.

Emphasized text is typed like this: \emph{this is emphasized}.
Bold       text is typed like this: \textbf{this is bold}.

\subsection{A Warning or Two}  % This command makes a subsection title.

If you get too much space after a mid-sentence period---abbreviations
like etc.\ are the common culprits)---then type a backslash followed by
a space after the period, as in this sentence.

Remember, don't type the 10 special characters (such as dollar sign and
backslash) except as directed!  The following seven are printed by
typing a backslash in front of them:  \$  \&  \#  \%  \_  \{  and  \}.  
The manual tells how to make other symbols.

%------------------------commented out!------------------------
%\end{document}                 % The input file ends with this command.
%------------------------commented out!------------------------

\noindent\rule{\textwidth}{0.8pt}

\setcounter{section}{\value{savesection}}
\setcounter{subsection}{\value{savesubsection}}

\caption{The result of processing the sample file}\label{fig:result}

\end{figure} % ----------------------------------------------------------------

\subsection{Running Text}

Most documents consist almost entirely of running text---words formed
into sentences, which are in turn formed into paragraphs---and the 
example file
is no exception. Describing running text poses no problems, you just type
it in naturally. In the output that it produces, \LaTeX{} will fill
lines and adjust the
spacing between words to give tidy left and right margins.
The spacing and distribution of the words in your input
file will have no effect at all on the eventual output.
Any number of spaces in your input file
are treated as a single space by \LaTeX{}, it also regards the
end of each line as a space between words \lls{17--19}.
A new paragraph is
indicated by a blank line in your input file, so don't leave
any blank lines unless you really wish to start a paragraph.

\LaTeX{} reserves a number of the less common keyboard characters for its
own use. The ten characters
\begin{quote}\begin{verbatim}
#  $  %  &  ~  _  ^  \  {  }
\end{verbatim}\end{quote}
should not appear as part of your text, because if they do
\LaTeX{} will get confused.

\subsection{\LaTeX{} Commands}

There are a number of words in the file that start `\verb|\|' \lls{11, 12 and
15}.  These are \LaTeX{} {\em commands\/} and they describe the structure
of your document.
There are a number of things that you should realise about these commands:
\begin{itemize}

\item All \LaTeX{} commands consist of a `\verb|\|' followed by one or more
characters.

\item \LaTeX{} commands should be typed using the correct mixture of upper- and
lower-case letters.  \verb|\BEGIN| is {\em not\/} the same as \verb|\begin|.

\item For some commands (such as \verb|\LaTeX| which produces ``\LaTeX'')
you just type the command.
There are other commands that look like
\begin{quote}\begin{verbatim}
\command{text}
\end{verbatim}\end{quote}
In this case the text is called the ``argument'' of the command.  The
\verb|\section| command is like this \llo{15}.
Sometimes you have to use curly brackets `\verb|{}|' to enclose the argument,
sometimes square brackets `\verb|[]|', and sometimes both at once.
There is method behind this apparent madness, but for the
time being you should be sure to copy the commands exactly as given.

\item When a command's name is made up entirely of letters, you must make sure
that the end of the command is marked by something that isn't a letter.
This is usually either the opening bracket around the command's argument, or
it's a space.  When it's a space, that space is always ignored by \LaTeX. We
will see later that this can sometimes be a problem.

\end{itemize}

\subsection{Overall structure}

There are some \LaTeX{} commands that must appear in every document.
The actual text of the document always starts with a
\verb|\begin{document}| command and ends with an \verb|\end{document}|
command \lls{11 and 40}.  Anything that comes after the
\verb|\end{document}| command is ignored.  Everything that comes
before the \verb|\begin{document}| command is called the
{\em preamble\/}. The preamble can only contain \LaTeX{} commands
to describe the document's style.

One command that must appear in the preamble is the
\verb|\documentclass| command \llo{11}.  This command specifies
the overall style for the document.  Our example file is a simple
technical document, and uses the \texttt{article} class, modified to
print in eleven-point type on A4 paper.
There are other classes and styles that you can use, as you will
find out later on in this document and elsewhere.

\subsection{Other Things to Look At}

\LaTeX{} can print both opening and closing quote characters, and can manage
either of these either single or double.  To do this it uses the two quote
characters from your keyboard: {\tt `} and {\tt '}. You will probably think of
{\tt '} as the ordinary single quote character which probably looks like
{\tt\symbol{'23}} or {\tt\symbol{'15}} on your keyboard,
and {\tt `} as a ``funny''
character that probably appears as {\tt\symbol{'22}}.
You type these characters once for single quote \llo{22},  and twice for
double quotes \llo{21}. The double quote character {\tt "} itself
is almost never used.

\LaTeX{} can produce three different kinds of dashes.
A long dash, for use as a punctuation symbol, as is typed as three dash
characters in a row, like this `\verb|---|' \llo{24}.  A shorter dash,
used between numbers as in `10--20', is typed as two dash
characters in a row, while a single dash character is used as a hyphen.

From time to time you will need to include one or more of the \LaTeX{}
special symbols in your text.  Seven of them can be printed by
making them into commands by proceeding them by backslash
\llo{37}.  The remaining three symbols can be produced by more
advanced commands, as can symbols that do not appear on your keyboard
such as \dag, \ddag, \S, \pounds, \copyright, $\sharp$ and $\clubsuit$.

It is sometimes useful to include comments in a \LaTeX{} file, to remind
you of what you have done or why you did it.  Everything to the
right of a \verb|%| sign is ignored by \LaTeX{}, and so it can
be used to introduce a comment.

\section{Document Classes and Options}\label{sec:styles}

There are four standard document classes available in \LaTeX:
\nobreak
\begin{description}

\item[\texttt{article}]  intended for short documents and articles for publication.
Articles do not have chapters, and when \verb|\maketitle| is used to generate
a title (see Section~\ref{sec:title}) it appears at the top of the first page
rather than on a page of its own.

\item[\texttt{report}] intended for longer technical documents.
It is similar to
{\tt article}, except that it contains chapters and the title appears on a page
of its own.

\item[\texttt{book}] intended as a basis for book publication.  Page layout is
adjusted assuming that the output will eventually be used to print on
both sides of the paper.

\item[\texttt{letter}]  intended for producing personal letters.  This style
will allow you to produce all the elements of a well laid out letter:
addresses, date, signature, etc.
\end{description}

These standard classes can be modified by a number of \emph{options}.
They appear in square brackets after the \verb|\documentclass| command.
Only one class can ever be used but you can have more than one option,
in which case their names should be separated by commas.  Some standard
options are:
\begin{description}

\item[\texttt{11pt}]  prints the document using eleven-point type for 
the running text
rather that the ten-point type normally used. Eleven-point type is about
ten percent larger than ten-point.

\item[\texttt{12pt}]  prints the document using twelve-point type for 
the running text
rather than the ten-point type normally used. Twelve-point type is about
twenty percent larger than ten-point.

\item[\texttt{a4paper}]  prints the document with margins 
suitable for A4 paper. \LaTeX{}
was designed in America where the standard paper is shorter and 
slightly wider than A4; without this option you will find that your 
output looks a little strange.

\item[\texttt{twoside}]  causes documents in the article or report styles to be
formatted for printing on both sides of the paper.  This is the default for the
book style.

\item[\texttt{twocolumn}] produces two column on each page.

\item[\texttt{titlepage}]  causes the \verb|\maketitle| command to generate a
title on a separate page for documents in the \fn{article} style.
A separate page is always used in both the \fn{report} and \fn{book} styles.

\end{description}

\section{Environments}

We mentioned earlier the idea of identifying a quotation to \LaTeX{} so that
it could arrange to typeset it correctly. To do this you enclose the
quotation between the commands \verb|\begin{quotation}| and
\verb|\end{quotation}|.
This is an example of a \LaTeX{} construction called an {\em environment\/}.
A number of
special effects are obtained by putting text into particular environments.

\subsection{Quotations}

There are two environments for quotations: \fn{quote} and \fn{quotation}.
\fn{quote} is used either for a short quotation or for a sequence of
short quotations separated by blank lines:
\egstart
\begin{verbatim}
US presidents ... pithy remarks:
\begin{quote}
The buck stops here;
I am not a crook\ldots
\end{quote}
and so on.
\end{verbatim}
\egmid%
US presidents have been known for their pithy remarks:
\begin{quote}
The buck stops here;
I am not a crook\ldots
\end{quote}
and so on.
\egend

Use the \fn{quotation} environment for quotations that consist of more
than one paragraph.  Paragraphs in the input are separated by blank
lines as usual:
\egstart
\begin{verbatim}
Here is some advice to remember:
\begin{quotation}
Environments for making
...other things as well.

Many problems
...environments.
\end{quotation}
\end{verbatim}
\egmid%
Here is some advice to remember:
\begin{quotation}
Environments for making quotations
can be used for other things as well.

Many problems can be solved by
novel applications of existing
environments.
\end{quotation}
\egend

\subsection{Centering and Flushing}

Text can be centred on the page by putting it within the \fn{center}
environment, and it will appear flush against the left or right margins if it
is placed within the \fn{flushleft} or \fn{flushright} environments.
Notice the
spelling of \fn{center}---unfortunately \LaTeX{} doesn't understand the English
spelling.

Text within these environments will be formatted in the normal way, in
{\samepage
particular the ends of the lines that you type are just regarded as spaces.  To
indicate a ``newline'' you need to type the \verb|\\| command.  For example:
\egstart
\begin{verbatim}
\begin{center}
one
two
three \\
four \\
five
\end{center}
\end{verbatim}
\egmid%
\begin{center}
one
two
three \\
four \\
five
\end{center}
\egend
}

\subsection{Lists}

There are three environments for constructing lists.  In each one each new
item is begun with an \verb|\item| command.  In the \fn{itemize} environment
the start of each item is given a marker, in the \fn{enumerate}
environment each item is marked by a number.  These environments can be nested
within each other in which case the amount of indentation used
is adjusted accordingly:
\egstart
\begin{verbatim}
\begin{itemize}
\item Itemized lists are handy.
\item However, don't forget
  \begin{enumerate}
  \item The `item' command.
  \item The `end' command.
  \end{enumerate}
\end{itemize}
\end{verbatim}
\egmid%
\begin{itemize}
\item Itemized lists are handy.
\item However, don't forget
  \begin{enumerate}
  \item The `item' command.
  \item The `end' command.
  \end{enumerate}
\end{itemize}
\egend

The third list making environment is \fn{description}.  In a description you
specify the item labels inside square brackets after the \verb|\item| command.
For example:
\egstart
\begin{verbatim}
Three animals that you should
know about are:
\begin{description}
  \item[gnat] A small animal...
  \item[gnu] A large animal...
  \item[armadillo] A ...
\end{description}
\end{verbatim}
\egmid%
Three animals that you should
know about are:
\begin{description}
  \item[gnat] A small animal that causes no end of trouble.
  \item[gnu] A large animal that causes no end of trouble.
  \item[armadillo] A medium-sized animal.
\end{description}
\egend

\subsection{Tables}

Because \LaTeX{} will almost always convert a sequence of spaces
into a single space,
it can be rather difficult to lay out tables.  See what happens in this example
\nolinebreak
\egstart
\begin{verbatim}
\begin{flushleft}
Income  Expenditure Result   \\
20s 0d  19s 11d     happiness \\
20s 0d  20s 1d      misery  \\
\end{flushleft}
\end{verbatim}
\egmid%
\begin{flushleft}
Income  Expenditure Result   \\
20s 0d  19s 11d     happiness \\
20s 0d  20s 1d      misery  \\
\end{flushleft}
\egend

The \fn{tabbing} environment overcomes this problem. Within it you set
tabstops and tab to them much like you do on a typewriter.  Tabstops are
set with the \verb|\=| command, and the \verb|\>| command moves to the
next stop.  The
\verb|\\| command is used to separate each line.  A line that ends \verb|\kill|
produces no output, and can be used to set tabstops:
\nolinebreak
\egstart
\begin{verbatim}
\begin{tabbing}
Income \=Expenditure \=    \kill
Income \>Expenditure \>Result \\
20s 0d \>19s 11d \>Happiness   \\
20s 0d \>20s 1d  \>Misery    \\
\end{tabbing}
\end{verbatim}
\egmid%
\begin{tabbing}
Income \=Expenditure \=    \kill
Income \>Expenditure \>Result \\
20s 0d \>19s 11d \>Happiness   \\
20s 0d \>20s 1d  \>Misery    \\
\end{tabbing}
\egend

Unlike a typewriter's tab key, the \verb|\>| command always moves to the next
tabstop in sequence, even if this means moving to the left.  This can cause
text to be overwritten if the gap between two tabstops is too small.

Another way of typesetting tables is to use the \verb|tabular|
environment.  For example,
\egstart
\begin{verbatim}
\begin{center}
\begin{tabular}{ll|c|c|r}
Surname & First name & Age & Sex & Treatment \\
\hline\hline
Regan & Ronald & 99 & male & Ignore \\
\hline
Major & John   & ? & indeterminate & Send to Coventry \\
\hline
Thatcher & Margaret & 134 & female & Shoot at dawn \\
\hline\hline
\end{tabular}
\end{center}
\end{verbatim}
\egmid%
\egend
gives the table
\begin{center}
\begin{tabular}{ll|c|c|r}
Surname & First name & Age & Sex & Treatment \\
\hline\hline
Regan & Ronald & 99 & male & Ignore \\
\hline
Major & John   & ? & indeterminate & Send to Coventry \\
\hline
Thatcher & Margaret & 134 & female & Shoot at dawn \\
\hline\hline
\end{tabular}
\end{center}

\subsection{Verbatim Output}

Sometimes you will want to include text exactly as it appears on a terminal
screen.  For example, you might want to include part of a computer program.
Not only do you want \LaTeX{} to stop playing around with the layout of your
text, you also want to be able to type all the characters on your keyboard
without confusing \LaTeX. The \fn{verbatim} environment has this effect:
\egstart
\begin{flushleft}
\verb|The section of program in|  \\
\verb|question is:|               \\
\verb|\begin{verbatim}|           \\
\verb|{ this finds %a & %b }|     \\[2ex]

\verb|for i := 1 to 27 do|        \\
\ \ \ \verb|begin|                \\
\ \ \ \verb|table[i] := fn(i);|   \\
\ \ \ \verb|process(i)|           \\
\ \ \ \verb|end;|                 \\
\verb|\end{verbatim}|
\end{flushleft}
\egmid%
The section of program in
question is:
\begin{verbatim}
{ this finds %a & %b }

for i := 1 to 27 do
   begin
   table[i] := fn(i);
   process(i)
   end;

\end{verbatim}
\egend

\section{Type Styles}

We have already come across the \verb|\emph| and \verb|\textbf|
commands for changing typeface.  
Here is a full list of the available typefaces.
\begin{quote}\small\begin{tabbing}
\verb|\textsc|~~\=\textsc{Small Caps}~%
 \=\verb|\textsc|~~\=\textsc{Small Caps}~%
 \=\verb|\textsc|~~\=\kill
\verb|\textrm|   \> \textrm{Roman}         \> \verb|\textit|   \> \textit{Italic}
                                  \> \verb|\textsc|   \> \textsc{Small Caps}    \\
\verb|\emph|   \> \emph{Emphatic}      \> \verb|\textsl| \> \textsl{Slanted}
                                  \> \verb|\texttt|   \> \texttt{Typewriter}     \\
\verb|\textbf|   \> \textbf{Boldface}      \> \verb|\textsf| \> \textsf{Sans Serif}
\end{tabbing}\end{quote}

Each of these commands take a single argument, the text to be affected.
Similar commands exist for changing typeface in \textit{maths mode}
(something you will come across later); these are
\verb|\mathrm|, \verb|\mathbf|, \verb|\mathit|, and \verb|\mathtt|.

For each of these one-argument commands, there is a version which 
affects \textit{all} subsequent text, and doesn't take any argument.  
These commands are \verb|\rm|, \verb|\it|, \verb|\sc|, \verb|\em|, 
\verb|\sl|, \verb|\tt|, \verb|\bf|, and \verb|\sf|, repectively,
and are intended for use with long sections of text.  (For short  
sections of text, you should always use \verb|\textit|, etc.)  
To use these commands you must limit the text to which they apply 
somehow or other.  It turns out that if you use them in an environment, 
as in
\egstart
\begin{verbatim}
Here is a slanted quotation.
\begin{quote}
\sl Play it again, Sam.
\end{quote}
This is commonly attributed to
Bogart.
\end{verbatim}
\egmid%
Here is a slanted quotation.
\begin{quote}
\sl Play it again, Sam.
\end{quote}
This is commonly attributed to
Bogart.
\egend
then their effect is turned off when the environment is
complete.  Another way to use them is within \verb|{| and \verb|}|,
as in 
\begin{quote}
\small\verb|{\sf Here's lookin' at you, kid.}|
\end{quote}
which gives {\sf Here's lookin' at you, kid} and returns to the
previous typeface.  Note too that the effect of these type changing
commands can vary depending on the document class you are working with.

In addition to the typeface commands, there are a set of 
commands that alter the size of the type.  These commands are:
\begin{quotation}\begin{tabbing}
\verb|\footnotesize|~~ \= \verb|\footnotesize|~~ \= \verb|\footnotesize| \=
 \kill
\verb|\tiny|           \> \verb|\small|          \> \verb|\large|        \>
\verb|\huge|  \\
\verb|\scriptsize|     \> \verb|\normalsize|     \> \verb|\Large|        \>
\verb|\Huge|  \\
\verb|\footnotesize|   \>                        \> \verb|\LARGE|
\end{tabbing}\end{quotation}
These take no arguments and 
are all used within braces or within some environment
so that their effect is limited to a single part of the text.

\section{Sectioning Commands and Tables of Contents}

Technical documents, like this one, are often divided into sections.
Each section has a heading containing a title and a number for easy
reference.  \LaTeX{} has a series of commands that will allow you to identify
different sorts of sections.  Once you have done this \LaTeX{} takes on the
responsibility of laying out the title and of providing the numbers.

The commands that you can use are:
\begin{quote}\begin{tabbing}
\verb|\subsubsection| \= \verb|\subsubsection|~~~~~~~~~~ \=           \kill
\verb|\chapter|       \> \verb|\subsection|    \> \verb|\paragraph|    \\
\verb|\section|       \> \verb|\subsubsection| \> \verb|\subparagraph|
\end{tabbing}\end{quote}
The naming of these last two is unfortunate, since they do not really have
anything to do with `paragraphs' in the normal sense of the word; they are just
lower levels of section.  In most document styles, headings made with
\verb|\paragraph| and \verb|\subparagraph| are not numbered.  \verb|\chapter|
is not available in document style \fn{article}.  The commands should be used
in the order given, since sections are numbered within chapters, subsections
within sections, etc.

A seventh sectioning command, \verb|\part|, is also available.  Its use is
always optional, and it is used to divide a large document into series of
parts.  It does not alter the numbering used for any of the other commands.

Including the command \verb|\tableofcontents| in your document will cause a
contents list to be included, containing information collected from the various
sectioning commands.  You will notice that each time your document is run
through \LaTeX{} the table of contents is always made up of the headings from
the previous version of the document.  This is because \LaTeX{} collects
information for the table as it processes the document, and then includes it
the next time it is run.  This can sometimes mean that the document has to be
processed through \LaTeX{} twice to get a correct table of contents.

\section{Producing Special Symbols}

You can include in your \LaTeX{} document a wide range of symbols that do not
appear on you your keyboard. For a start, you can add an accent to any letter:
\begin{quote}\begin{tabbing}

\t{oo} \= \verb|\t{oo}|~~~ \=
\t{oo} \= \verb|\t{oo}|~~~ \=
\t{oo} \= \verb|\t{oo}|~~~ \=
\t{oo} \= \verb|\t{oo}|~~~ \=
\t{oo} \= \verb|\t{oo}|~~~ \=
\t{oo} \=                       \kill

\a`{o} \> \verb|\`{o}|  \> \~{o}  \> \verb|\~{o}|  \> \v{o}  \> \verb|\v{o}| \>
\c{o}  \> \verb|\c{o}|  \> \a'{o} \> \verb|\'{o}|  \\
\a={o} \> \verb|\={o}|  \> \H{o}  \> \verb|\H{o}|  \> \d{o}  \> \verb|\d{o}| \>
\^{o}  \> \verb|\^{o}|  \> \.{o}  \> \verb|\.{o}|  \\
\t{oo} \> \verb|\t{oo}| \> \b{o}  \> \verb|\b{o}|  \\  \"{o} \> \verb|\"{o}| \>
\u{o}  \> \verb|\u{o}|  \\
\end{tabbing}\end{quote}

A number of other symbols are available, and can be used by including the
following commands:
\begin{quote}\begin{tabbing}

\LaTeX~\= \verb|\copyright|~~~~ \= \LaTeX~\= \verb|\copyright|~~~~ \=
\LaTeX~\=  \kill

\dag       \> \verb|\dag|       \> \S     \> \verb|\S|     \>
\copyright \> \verb|\copyright| \\
\ddag      \> \verb|\ddag|      \> \P     \> \verb|\P|     \>
\pounds    \> \verb|\pounds|    \\
\oe        \> \verb|\oe|        \> \OE    \> \verb|\OE|    \>
\ae        \> \verb|\AE|        \\
\AE        \> \verb|\AE|        \> \aa    \> \verb|\aa|    \>
\AA        \> \verb|\AA|        \\
\o         \> \verb|\o|         \> \O     \> \verb|\O|     \>
\l         \> \verb|\l|         \\
\L         \> \verb|\E|         \> \ss    \> \verb|\ss|    \>
?`         \> \verb|?`|         \\
!`         \> \verb|!`|         \> \ldots \> \verb|\ldots| \>
\LaTeX     \> \verb|\LaTeX|     \\
\end{tabbing}\end{quote}
There is also a \verb|\today| command that prints the current date. When you
use these commands remember that \LaTeX{} will ignore any spaces that
follow them, so that you can type `\verb|\pounds 20|' to get `\pounds 20'.
However, if you type `\verb|\LaTeX is wonderful|' you will get `\LaTeX is
wonderful'---notice the lack of space after \LaTeX.
To overcome this problem you can follow any of these commands by a
pair of empty brackets and then any spaces that you wish to include,
and you will see that
\verb|\LaTeX{} \texit{is} wonderful!| (\LaTeX{} \textit{is} wonderful!).

Finally, \LaTeX{} `math' mode, normally used to layout mathematical
formulae, gives access to an even larger range of symbols, including the
upper and lower case greek mathematical alphabet, calligraphic letters,
mathematical operators and relations, arrows and a whole lot more.  This
document has not mentioned math mode, and it isn't going to either, so you
will have to refer to the manual if you really need something that you
can't get otherwise.

\section{Titles}\label{sec:title}

Most documents have a title.  To title a \LaTeX{} document, you include the
following commands in your document, usually just after
\verb|\begin{document}|.
\begin{quote}\footnotesize\begin{verbatim}
\title{required title}
\author{required author}
\date{required date}
\maketitle
\end{verbatim}\end{quote}
If there are several authors, then their names should be separated by
\verb|\and|; they can also be separated by \verb|\\| if you want them to be
centred on different lines.  If the \verb|\date| command is left out, then the
current date will be printed.
\egstart
\begin{verbatim}
\title{Essential \LaTeX}
\author{J Warbrick\and A N Other}
\date{14th February 1988}
\maketitle
\end{verbatim}
\egmid
\begin{center}
{\normalsize Essential \LaTeX}\\[4ex]
J Warbrick\hspace{1em}A N Other\\[2ex]
14th February 1988
\end{center}
\egend

The exact appearance of the title varies depending on
the document style.  In styles \fn{report} and \fn{book} the title appears on a
page of its own. In the \fn{article} style it normally appears at the top
of the first page, the style option \fn{titlepage} will alter this (see
Section~\ref{sec:styles}).

\let\oltableofcontents\tableofcontents
\let\tableofcontents\relax
\let\listoftables\relax
\let\listoffigures\relax

% This is a supplement to:

% Essential LaTeX - Jon Warbrick 02/88
%                 - W T Hewitt, MCC CGU August 1989

% giving a brief overview of mathematical typesetting using LaTeX
%                      D. P. Carlisle 22/9/89

% modified by Richard Kaye for LaTeX2e

% I reproduce here the copyright notice from Essential LaTeX, the same
% conditions apply to this document.

%   Permission is granted to reproduce the document in any way providing
%   that it is distributed for free, except for any reasonable charges for
%   printing, distribution, staff time, etc.  Direct commercial
%   exploitation is not permitted.  Extracts may be made from this
%   document providing an acknowledgement of the original source is
%   maintained.


\documentclass[a4paper]{article}
\usepackage{latexsym}
\usepackage{amssymb}
\usepackage{mathrsfs}
% 
\newcounter{savesection}
\newcounter{savesubsection}

% commands to do 'LaTeX Manual-like' examples

\newlength{\egwidth}\setlength{\egwidth}{0.45\textwidth}

\newenvironment{eg}%
{\begin{list}{}{\setlength{\leftmargin}{0pt}%
\setlength{\rightmargin}{\leftmargin}}\item[]\footnotesize}%
{\end{list}}

% AmsTeX logo
\newcommand{\AmsTeX}{{$\cal A$}\kern-.1667em\lower.5ex\hbox
{$\cal M$}\kern-.125em{$\cal S$}-\TeX}
\newcommand{\AmsLaTeX}{{$\cal A$}\kern-.1667em\lower.5ex\hbox
{$\cal M$}\kern-.125em{$\cal S$}-\LaTeX}
%
\newenvironment{egbox}%
{\begin{minipage}[t]{\egwidth}}%
{\end{minipage}}

\newcommand{\egstart}{\begin{eg}\begin{egbox}}
\newcommand{\egmid}{\end{egbox}\hfill\begin{egbox}}
\newcommand{\egend}{\end{egbox}\end{eg}}

% one or two other commands

\newcommand{\fn}[1]{\hbox{\tt #1}}
\newcommand{\llo}[1]{(see line #1)}
\newcommand{\lls}[1]{(see lines #1)}
\newcommand{\bs}{$\backslash$}

\title{Essential Mathematical \LaTeXe}
\author{D. P. Carlisle \and Richard Kaye}
\date{12th August 1998}

\begin{document}

\maketitle

\section{Introduction}
This document is a supplement to ``Essential \LaTeXe'', describing
basic mathematical typesetting in \LaTeXe. It makes no attempt at 
completeness so if in doubt {\bf read the manual}.

\section{Math, Display-math and Equation}
Mathematics is treated by \LaTeX\  completely differently from ordinary text.
There are two special {\em modes\/} for mathematics, {\em math mode\/}
and {\em display math mode}.

Math mode commands are surrounded by \verb|\(| and \verb|\)|.
\egstart
Some mathematics set inline \( 2\times 3 =   6 \).
Note that spaces in the input file are ignored in math mode.
\egmid
\begin{verbatim}
... set inline \( 2\times 3 =   6 \).
Note that spaces ...
\end{verbatim}
\egend

Display math mode commands are surrounded by \verb|\[| and \verb|\]|.
\egstart
A larger equation to be displayed on a line by itself.
\[ f(x) = \sum_{i=0}^{\infty}\frac{f^{(i)}(x)}{i!} \]
\egmid
\begin{verbatim}
... to be displayed on a line by itself.
\[ f(x) = 
 \sum_{i=0}^{\infty}\frac{f^{(i)}(x)}{i!} \]
\end{verbatim}
\egend

There is a variant of Display math mode, the equation environment
which automatically generates an equation number.

\egstart
\begin{equation}
\left(\begin{array}{cc}
1 & 2 \\0 & 1
\end{array}\right)
\left(\begin{array}{cc}
2 & 0\\1 & 3
\end{array}\right)
=
\left(\begin{array}{cc}
4 & 6 \\1 & 3
\end{array}\right)
\end{equation}
\egmid
\begin{verbatim}
\begin{equation}
\left(\begin{array}{cc}
1 & 2 \\0 & 1
\end{array}\right)
\left(\begin{array}{cc}
2 & 0\\1 & 3
\end{array}\right)
=
\left(\begin{array}{cc}
4 & 6 \\1 & 3
\end{array}\right)
\end{equation}
\end{verbatim}
\egend

The short examples above show the main types of commands available 
in math mode.
\begin{enumerate}
\item Subscripts and superscripts are produced with \verb|_| and \verb|^|, as in
      \verb|x_{1} = p^{2}| \(x_{1} = p^{2}\). 
\item Fractions are produced by the \verb|\frac| command. 
      \verb|\(\frac{a + b}{c}\)| \(\frac{a + b}{c}\).
\item Various commands give names to mathematical symbols.\\
      \verb| \infty \Rightarrow \surd \bigotimes | 
      \(\infty\;\Rightarrow\;\surd\;\bigotimes\)
\item Arrays are produced by the array environment. This is identical
      to the tabular environment described in {\em essential \LaTeX\/}
      except that the entries are typeset in math mode instead of LR mode.
      Note that the array environment does not put brackets arround the array
      so it can also be used for setting determinants, or even 
      sets of equations in which you want the columns to line up.
\item The commands \verb|\left| and \verb|\right| produce delimiters which
      grow as large as needed, they can be used with a variety of symbols,
      e.g., \verb|\left(|, \verb|\left\{|, \verb!\left|!. The full set of 
      these delimiters is shown in Table \ref{dels} below.  There are
      some things to note: 
      \begin{enumerate}
      \item \verb|\left| and \verb|\right| must come in matching
      pairs, otherwise \LaTeX\ won't know what formula the delimiter
      must be adjusted to fit round; 
      \item you must remember to put a delimiter after \verb|\left| 
      and \verb|\right|; 
      \item the delimiters themselves needn't be matched, so 
      \verb|\( \left( ... \right] \)|
      is OK; 
      \item if you \textit{really} don't want any delimiter on
      one side, use `.' in place of a delimiter, as in 
      \verb|\( \left\{ ... \right. \)| (which is often used for definitions
      by cases).
      \end{enumerate}
\end{enumerate}

\section{Spacing}
All spaces in the input file are ignored in math mode. Sometimes
you may want to adjust the spacing. Use one of the following commands
or an explicit \verb|\hspace|.
\begin{center}
\begin{tabular}{ll}
\verb|\,| thin space          &\verb|\:| medium space \\
\verb|\!| negative thin space &\verb|\;| thick space \\
\end{tabular}
\end{center}

A good example of where \LaTeX\ needs some help with spacing is
\egstart
$\int\!\!\int z\, dx dy$ \ instead of $\int\int z dx dy$
\egmid
\verb|\int\!\!\int z\, dx dy .. \int\int z dx dy|.
\egend

\section{Changing fonts in math mode}
The default math mode font is $math\ italic$. This should not be
confused with ordinary {\em text italic}.  The letters are
a slightly different shape, and the spacing between them is 
quite different.
The font for `ordinary' letters can be changed with the 
commands, \verb|\mathbf|, \verb|\mathrm|, \verb|\mathit|,
\verb|\mathsf|, and \verb|\mathnormal|.
These commands take a single argument---the letters that
should be bold or roman or whatever.
Note that lower case Greek letters
are just regarded as mathematical symbols, \verb|\alpha| etc, and are
not affected by these commands.

\verb|\bf| produces {\bf bold face roman} letters. If you wish
to have {\boldmath $bold$ $face$ $math$ $italic$} letters, and bold face 
Greek letters and mathematical symbols, use the \verb|\boldmath| command
{\em before\/} going into math mode. 
This changes the default math fonts to bold.

\egstart{\normalsize
\( x = 2\pi \Rightarrow x \simeq 6.28 \)\\
\(\mathbf{ x = 2\pi \Rightarrow x \simeq 6.28 } \)\\\mbox{}\\
\boldmath
\( x = 2\pi \Rightarrow x \simeq 6.28 \)}
\egmid
\begin{verbatim}
\( x = 2\pi \Rightarrow x \simeq 6.28 \)
\(\mathbf{ x = 2\pi \Rightarrow x \simeq 6.28 } \)
\boldmath
\( x = 2\pi \Rightarrow x \simeq 6.28 \)
\end{verbatim}
\egend

There is also a calligraphic font for upper case letters
produced by the \verb|\mathcal| command.

\egstart
\( \mathcal{F}\mathcal{G} \)
\egmid
\verb|\( \mathcal{F}\mathcal{G} \)|
\egend

\AmsLaTeX\ provides more commands for bold-face mathematics, so that
bold math italic and ordinary math fonts can be mixed in the same
formula without using \verb|\boldmath|, and it also adds `blackboard bold'
characters (\verb|\mathbb|), $\mathbb{NZQRC}$, 
`script', and `fraktur' fonts (\verb|\mathfrak|). $\mathfrak{ABCDEFG}$.  
(RK adds: I'm not very fond of the AMS's script
font though.  I much prefer the oldfashioned copper-plate letters
which are available in the `mathrsfs' package, and which gives---using the
\verb|\mathscr| command---$\mathscr{ABCDEFG}$.)

\section{Symbols}
The tables show most of the symbols available from the standard \LaTeX\ symbol 
fonts. Negations of the relational symbols can be made with the \verb|\not| 
command.

\egstart
$ G \not\equiv H $ 
\egmid
\verb|G \not\equiv H|
\egend

A very common mistake is to ignore the difference between
the various `kinds' of symbols.  If you check the tables, you'll
notice that some symbols are available in different ways.  The 
difference is the spacing on either side of the symbol.
For example to get a colon in math mode, you type \verb|:| or
\verb|\colon|.  The second is a puctuation symbol, the first is 
a relation symbol.  Here are examples of their \textit{proper} use.
\egstart
\( f\colon A\to B \)\\
\( \{ x : x \not\in x \} \)
\egmid
\verb|\( f\colon A\to B \)|
\verb|\( \{ x : x \not\in x \} \)|
\egend

The standard \LaTeX\ fonts have enough symbols for most people, see
Tables~1--\ref{other}.
If you need more you could try the \AmsTeX\ extra symbol fonts, most easily
obtained by loading \verb|amssymb| using the \verb|\usepackage| command.
See Tables \ref{ams}~to \ref{bbams}.  Much more control over mathematical
typesetting is available as a series of \AmsLaTeX{} packages you can
also load.

\section{What about \$'s?}
If you have converted to \LaTeX\ from plain or \AmsTeX, you will
probably be wondering why there has been no mention of \verb|$| 
and \verb|$$|.

In these systems math mode is surrounded by \verb|$|'s and display math mode
is surrounded by \verb|$$|. This has certain drawbacks over the \LaTeX\ system
as it is difficult for your text editor to match \verb|$|'s as it is hard to 
tell which ones are starting math mode and which are ending it. \TeX\ will also
get confused if you miss a \$ out.

The (incorrect) input 
\begin{verbatim}
let (a,b,c)$ be a Pythagorean triple, i.e.\ three 
integers such that $a^{2}+b^{2}=c^{2}$.
\end{verbatim}
produces the slightly mysterious error message
\begin{verbatim}
! Missing $ inserted.
<inserted text> 
		$
<to be read again> 
		   ^
l.56 ...triple, i.e.\ three integers such that $a^
						  {2}+b^{2}=c^{2}$
? 
\end{verbatim}

Note that it reports the wrong error and in the wrong place, the use of the
\verb|^| command out of math mode. \TeX\ has typeset `be a \ldots\ such that'
in math mode and {\em exited\/} math mode at the \verb|$| after `such that'. 
If you had made the equivalent \LaTeX\ error, \LaTeX\ 
has a better idea of what you indended:
\begin{verbatim}
let (a,b,c)\) be a Pythagorean triple, i.e.\ three 
integers such that \(a^{2}+b^{2}=c^{2}\)
\end{verbatim}
The error message may still be unintelligable, but at least
it reports the error in the right place, you have used \verb|\)|
to end math mode when you were not in math mode (as you omitted the 
\verb|\(| which should have been before the {\tt (a,b,c)}).
\begin{verbatim}
LaTeX error.  See LaTeX manual for explanation.
	      Type  H <return>  for immediate help.
! Bad math environment delimiter.
\@latexerr ...for immediate help.}\errmessage {#1}
						  
\)...ifinner $\else \@badmath \fi \else \@badmath 
						  \fi 
l.56 let (a,b,c)\)
		   be a Pythagorean triple, i.e.\ three integers such that \...

? 
\end{verbatim}

The single dollar is sometimes useful for small
sections of math.
\egstart
Let $G$ be a $p$-group
\egmid
\verb|Let $G$ be a $p$-group|
\egend

The double dollar is {\em not\/} always equivalent to \verb|\[ ... \]|,
and so should not be used if you want your \LaTeX\ file to be compatible
with different styles and style options (try the fleqn style option).

\clearpage

\def\W#1#2{$#1{#2}$ &\tt\string#1\string{#2\string}}
\def\X#1{$#1$ &\tt\string#1}
\def\Y#1{$\big#1$ &\tt\string#1}
\def\Z#1{\tt\string#1}


\begin{table}[p]
\centering
\begin{tabular}{*8l}
\X\alpha        &\X\theta       &\X o           &\X\tau         \\
\X\beta         &\X\vartheta    &\X\pi          &\X\upsilon     \\
\X\gamma        &\X\gamma       &\X\varpi       &\X\phi         \\
\X\delta        &\X\kappa       &\X\rho         &\X\varphi      \\
\X\epsilon      &\X\lambda      &\X\varrho      &\X\chi         \\
\X\varepsilon   &\X\mu          &\X\sigma       &\X\psi         \\
\X\zeta         &\X\nu          &\X\varsigma    &\X\omega       \\
\X\eta          &\X\xi                                          \\
								\\
\X\Gamma        &\X\Lambda      &\X\Sigma       &\X\Psi         \\
\X\Delta        &\X\Xi          &\X\Upsilon     &\X\Omega       \\
\X\Theta        &\X\Pi          &\X\Phi
\end{tabular}
\caption{Greek Letters}\label{greek}
\end{table}

\begin{table}
\centering
\begin{tabular}{*8l}
\X\pm           &\X\cap         &\X\diamond             &\X\oplus       \\
\X\mp           &\X\cup         &\X\bigtriangleup       &\X\ominus      \\
\X\times        &\X\uplus       &\X\bigtriangledown     &\X\otimes      \\
\X\div          &\X\sqcap       &\X\triangleleft        &\X\oslash      \\
\X\ast          &\X\sqcup       &\X\triangleright       &\X\odot        \\
\X\star         &\X\vee         &\X\bigcirc     &\X\amalg \\
\X\circ         &\X\wedge       &\X\dagger      &\X\wr      \\
\X\bullet       &\X\setminus    &\X\ddagger     &\X\cdot    \\
\X+             &\X-
\end{tabular}
\caption{Binary Operation Symbols}\label{bin}
\end{table}


\begin{table}
\centering
\begin{tabular}{*8l}
\X\leq          &\X\geq         &\X\equiv       &\X\models      \\
\X\prec         &\X\succ        &\X\sim         &\X\perp        \\
\X\preceq       &\X\succeq      &\X\simeq       &\X\mid         \\
\X\ll           &\X\gg          &\X\asymp       &\X\parallel    \\
\X\subset       &\X\supset      &\X\approx      &\X\bowtie      \\
\X\subseteq     &\X\supseteq    &\X\cong        &\X\propto      \\
\X\neq          &\X\smile       &\X\vdash       &\X\dashv       \\
\X\sqsubseteq   &\X\sqsupseteq  &\X\doteq       &\X\frown       \\
\X\in           &\X\ni          &\X=            &\X>            \\
\X<             &$:$&:
\end{tabular}
\caption{Relation Symbols}\label{rel}
\end{table}

\begin{table}
\centering
\begin{tabular}{*{5}{lp{3.2em}}}
\X,     &\X;    &\X\colon       &\X\ldotp       &\X\cdotp
\end{tabular}
\caption{Punctuation Symbols}\label{punct}
\end{table}

\begin{table}
\centering
\begin{tabular}{*6l}
\X\leftarrow            &\X\longleftarrow       &\X\uparrow     \\
\X\Leftarrow            &\X\Longleftarrow       &\X\Uparrow     \\
\X\rightarrow           &\X\longrightarrow      &\X\downarrow   \\
\X\Rightarrow           &\X\Longrightarrow      &\X\Downarrow   \\
\X\leftrightarrow       &\X\longleftrightarrow  &\X\updownarrow \\
\X\Leftrightarrow       &\X\Longleftrightarrow  &\X\Updownarrow \\
\X\mapsto               &\X\longmapsto          &\X\nearrow     \\
\X\hookleftarrow        &\X\hookrightarrow      &\X\searrow     \\
\X\leftharpoonup        &\X\rightharpoonup      &\X\swarrow     \\
\X\leftharpoondown      &\X\rightharpoondown    &\X\nwarrow
\end{tabular}
\caption{Arrow Symbols}
\end{table}

\begin{table}
\centering
\begin{tabular}{*8l}
\X\ldots        &\X\cdots       &\X\vdots       &\X\ddots       \\
\X\aleph        &\X\prime       &\X\forall      &\X\infty       \\
\X\hbar         &\X\emptyset    &\X\exists      &\X\spadesuit   \\
\X\imath        &\X\nabla       &\X\neg         &\X\heartsuit   \\
\X\jmath        &\X\surd        &\X\flat        &\X\diamondsuit \\
\X\ell          &\X\top         &\X\natural     &\X\clubsuit    \\
\X\wp           &\X\bot         &\X\sharp       &\X\partial     \\
\X\Re           &\X\|           &\X\backslash   &\X\triangle    \\
\X\Im           &\X\angle       &\X.            &\X|
\end{tabular}
\caption{Miscellaneous Symbols}\label{ord}
\end{table}

\begin{table}
\centering
\begin{tabular}{*6l}
\X\sum          &\X\bigcap      &\X\bigodot     \\
\X\prod         &\X\bigcup      &\X\bigotimes   \\
\X\coprod       &\X\bigsqcup    &\X\bigoplus    \\
\X\int          &\X\bigvee      &\X\biguplus    \\
\X\oint         &\X\bigwedge    
\end{tabular}
\caption{Variable-sized  Symbols}\label{op}
\end{table}


\begin{table}
\centering
\begin{tabular}{*8l}
\Z\arccos &\Z\cos  &\Z\csc &\Z\exp &\Z\ker    &\Z\limsup &\Z\min &\Z\sinh \\
\Z\arcsin &\Z\cosh &\Z\deg &\Z\gcd &\Z\lg     &\Z\ln     &\Z\Pr  &\Z\sup  \\
\Z\arctan &\Z\cot  &\Z\det &\Z\hom &\Z\lim    &\Z\log    &\Z\sec &\Z\tan  \\
\Z\arg    &\Z\coth &\Z\dim &\Z\inf &\Z\liminf &\Z\max    &\Z\sin &\Z\tanh
\end{tabular}
\caption{Log-like Symbols}\label{log}
\end{table}


\begin{table}
\centering
\begin{tabular}{*8l}
\X(             &\X)            &\X\uparrow     &\X\Uparrow     \\
\X[             &\X]            &\X\downarrow   &\X\Downarrow   \\
\X\{            &\X\}           &\X\updownarrow &\X\Updownarrow \\
\X\lfloor       &\X\rfloor      &\X\lceil       &\X\rceil       \\
\X\langle       &\X\rangle      &\X/            &\X\backslash   \\
\X|             &\X\|
\end{tabular}
\caption{Delimiters\label{dels}}
\end{table}

\begin{table}
\centering
\begin{tabular}{*8l}
\Y\rmoustache&  \Y\lmoustache&  \Y\rgroup&      \Y\lgroup\\[5pt]
\Y\arrowvert&   \Y\Arrowvert&   \Y\bracevert
\end{tabular}
\caption{Large Delimiters\label{ldels}}
\end{table}

\begin{table}
\centering
\begin{tabular}{*{10}l}
\W\hat{a}       &\W\acute{a}    &\W\bar{a}      &\W\dot{a}      &\W\breve{a}    \\
\W\check{a}     &\W\grave{a}    &\W\vec{a}      &\W\ddot{a}     &\W\tilde{a}    \\
\end{tabular}
\caption{Math mode accents}\label{accent}
\end{table}

\begin{table}
\centering
\begin{tabular}{*4l}
\W\widetilde{abc}       &\W\widehat{abc}                \\
\W\overleftarrow{abc}   &\W\overrightarrow{abc}         \\
\W\overline{abc}        &\W\underline{abc}              \\
\W\overbrace{abc}       &\W\underbrace{abc}             \\[5pt]
\W\sqrt{abc}            &$\sqrt[n]{abc}$&\verb|\sqrt[n]{abc}|\\
$f'$&\verb|f'|          &$\frac{abc}{xyz}$&\verb|\frac{abc}{xyz}|
\end{tabular}
\caption{Some other constructions}
\end{table}

\begin{table}
\centering
\begin{tabular}{*4l}
\X\mho             &\X\Join            \\
\X\Box             &\X\Diamond         \\
\X\sqsubset        &\X\sqsupset        \\
\X\lhd             &\X\rhd             \\
\X\unlhd           &\X\unrhd           \\
\X\leadsto
\end{tabular}
\caption{Further symbols available with the \texttt{latexsym} package}\label{other}
\end{table}

\begin{table}
\centering
\begin{tabular}{*4l}
\X\boxdot               &\X\boxplus \\ 
	\X\boxtimes            &\X\square              \\
\X\blacksquare          &\X\centerdot\\ 
	\X\lozenge             &\X\blacklozenge        \\
\X\circlearrowright     &\X\circlearrowleft\\ 
	\X\rightleftharpoons   &\X\leftrightharpoons   \\
\X\boxminus             &\X\Vdash\\ 
	\X\Vvdash              &\X\vDash               \\
\X\twoheadrightarrow    &\X\twoheadleftarrow\\ 
	\X\leftleftarrows      &\X\rightrightarrows    \\
\X\upuparrows           &\X\downdownarrows\\ 
	\X\upharpoonright      &\X\downharpoonright    \\
\X\upharpoonleft        &\X\downharpoonleft\\ 
	\X\rightarrowtail      &\X\leftarrowtail       \\
\X\leftrightarrows      &\X\rightleftarrows\\ 
	\X\Lsh                 &\X\Rsh                 \\
\end{tabular}
\caption{Some AMS symbols\label{ams}}
\end{table}

\begin{table}
\centering
\begin{tabular}{*4l}
\X\rightsquigarrow      &\X\leftrightsquigarrow\\ 
	\X\looparrowleft       &\X\looparrowright      \\
\X\circeq               &\X\succsim\\ 
	\X\gtrsim              &\X\gtrapprox           \\
\X\multimap             &\X\therefore\\ 
	\X\because             &\X\doteqdot            \\
\X\triangleq            &\X\precsim\\ 
	\X\lesssim             &\X\lessapprox          \\
\X\eqslantless          &\X\eqslantgtr\\ 
	\X\curlyeqprec         &\X\curlyeqsucc         \\
\X\preccurlyeq          &\X\leqq\\ 
	\X\leqslant            &\X\lessgtr             \\
\X\backprime            &\X\risingdotseq\\ 
	\X\fallingdotseq       &\X\succcurlyeq         \\
\X\geqq                 &\X\geqslant\\ 
	\X\gtrless             &\X\sqsubset            \\
\X\sqsupset             &\X\vartriangleright\\ 
	\X\vartriangleleft     &\X\trianglerighteq     \\
\X\trianglelefteq       &\X\bigstar\\ 
	\X\between             &\X\blacktriangledown   \\
\X\blacktriangleright   &\X\blacktriangleleft\\ 
	\X\vartriangle         &\X\blacktriangle       \\
\X\triangledown         &\X\eqcirc\\ 
	\X\lesseqgtr           &\X\gtreqless           \\
\X\lesseqqgtr           &\X\gtreqqless\\ 
	\X\Rrightarrow         &\X\Lleftarrow          \\
\X\veebar               &\X\barwedge\\ 
	\X\doublebarwedge      &\X\angle               \\
\X\measuredangle        &\X\sphericalangle\\ 
	\X\varpropto           &\X\smallsmile          \\
\X\smallfrown           &\X\Subset\\ 
	\X\Supset              &\X\Cup                 \\
\X\Cap                  &\X\curlywedge\\ 
	\X\curlyvee            &\X\leftthreetimes      \\
\X\rightthreetimes      &\X\subseteqq\\ 
	\X\supseteqq           &\X\bumpeq              \\
\X\Bumpeq               &\X\lll\\ 
	\X\ggg                 &\X\circledS            \\
\X\pitchfork            &\X\dotplus\\ 
	\X\backsim             &\X\backsimeq           \\
\X\complement           &\X\intercal\\ 
	\X\circledcirc         &\X\circledast          \\
\X\circleddash          &\X\ulcorner\\ 
	\X\urcorner            &\X\llcorner            \\
\X\lrcorner             &\X\yen\\ 
	\X\checkmark           &\X\circledR            \\
\X\maltese              &\X\lvertneqq\\ 
	\X\gvertneqq           &\X\nleq                \\
\end{tabular}
\caption{More AMS symbols\label{mams}}
\end{table}

\begin{table}
\centering
\begin{tabular}{*4l}

\X\ngeq                 &\X\nless\\ 
	\X\ngtr                &\X\nprec               \\
\X\nsucc                &\X\lneqq\\ 
	\X\gneqq               &\X\nleqslant           \\
\X\ngeqslant            &\X\lneq\\ 
	\X\gneq                &\X\npreceq             \\
\X\nsucceq              &\X\precnsim\\ 
	\X\succnsim            &\X\lnsim               \\
\X\gnsim                &\X\nleqq\\ 
	\X\ngeqq               &\X\precneqq            \\
\X\succneqq             &\X\precnapprox\\ 
	\X\succnapprox         &\X\lnapprox            \\
\X\gnapprox             &\X\nsim\\ 
	\X\ncong               &\X\varsubsetneq        \\
\X\varsupsetneq         &\X\nsubseteqq\\ 
	\X\nsupseteqq          &\X\subsetneqq          \\
\X\supsetneqq           &\X\varsubsetneqq\\ 
	\X\varsupsetneqq       &\X\subsetneq           \\
\X\supsetneq            &\X\nsubseteq\\ 
	\X\nsupseteq           &\X\nparallel           \\
\X\nmid                 &\X\nshortmid\\ 
	\X\nshortparallel      &\X\nvdash              \\
\X\nVdash               &\X\nvDash\\ 
	\X\nVDash              &\X\ntrianglerighteq    \\
\X\ntrianglelefteq      &\X\ntriangleleft\\ 
	\X\ntriangleright      &\X\nleftarrow          \\
\X\nrightarrow          &\X\nLeftarrow\\ 
	\X\nRightarrow         &\X\nLeftrightarrow     \\
\X\nleftrightarrow      &\X\divideontimes\\ 
	\X\varnothing          &\X\nexists             \\
\X\mho                  &\\% was: \X\thorn\\ 
	\X\beth                &\X\gimel               \\
\X\daleth               &\X\lessdot\\ 
	\X\gtrdot              &\X\ltimes              \\
\X\rtimes               &\X\shortmid\\ 
	\X\shortparallel       &\X\smallsetminus       \\
\X\thicksim             &\X\thickapprox\\ 
	\X\approxeq            &\X\succapprox          \\
\X\precapprox           &\X\curvearrowleft\\ 
	\X\curvearrowright     &\X\digamma             \\
\X\varkappa             &\X\hslash\\ 
	\X\hbar                &\X\backepsilon
\end{tabular}
\caption{Even more AMS symbols\label{emams}}
\end{table}

\begin{table}
\centering
\begin{tabular}{*2l}
$\mathbb{A~B~C~D~E~F}$ & \verb|\mathbb A \mathbb B ... |\\
$\mathfrak{A~B~C~D~E~F}$ & \verb|\mathfrak A \mathfrak B ... |\\
$\mathscr{A~B~C~D~E~F}$ &\verb|\mathscr A \mathscr B ... |
\end{tabular}
\caption{Other symbol fonts available with \texttt{amssymb} and \texttt{mathrsfs}}\label{bbams}
\end{table}



\end{document}

\documentclass[a4paper]{article}

\newtheorem{thm}{Theorem}[section]
\newtheorem{chall}[thm]{Challenge}

\title{Cross references in \LaTeX}
\author{Richard Kaye\\School of Mathematics and Statistics\\
The University of Birmingham\\Birmingham B15 2TT\\U.K.}

\date{12th October 2000}

\begin{document}

\maketitle

% any of the following commands may be used to give the title
% and list of contents/tables/figures

\tableofcontents % writes and/or reads in crossref.toc
\listoftables    % writes and/or reads in crossref.lot
\listoffigures   % writes and/or reads in crossref.lof

\section{Introduction}

This document is intended to be used as a model for
\LaTeX\ documents that use cross-referencing.  It should
be studied in conjunction with its source the file, 
\texttt{crossref.tex}, which was used to produce it.  Only 
the most basic kinds of \LaTeX~cross-referencing commands 
are mentioned here. In particular, I will not mention 
indexing nor creating a glossary or index of notation. 
Perhaps the most 
important omission here is  Bib\TeX, for bibliographies.
All these tools are discussed and used in other example
documents,~\cite{indexntn,extraref} available on the web.

Note that the parts labelled `challenge' (such as 
Challenge~\ref{doesnotconverge} on 
page~\pageref{doesnotconverge} below) are intended 
for experts only.  They are \textit{not} easy.

\section{More cross references}\label{sec:more}

To show how sections can be
labelled and referred to, note that this is section~\ref{sec:more}.

This section also contains a more complicated example, a figure
created with the \LaTeX\ `picture' environment (Figure~\ref{samplefigure}).

\begin{figure}
\caption{A figure by Lamport}\label{samplefigure}
\begin{center}
%this Example is from Lamport's book on LaTeX:
\newcounter{cms}
\setlength{\unitlength}{1mm}% set units to 1mm
\begin{picture}(50,39)% size in units of 1mm
\put(0,7){\makebox(0,0)[bl]{cm}}
\multiput(10,7)(10,0){5}{\addtocounter{cms}{1}\makebox(0,0)[b]{\arabic{cms}}}
\put(15,20){\circle{6}}
\put(30,20){\circle{6}}
\put(15,20){\circle*{2}}
\put(30,20){\circle*{2}}
\put(10,24){\framebox(25,8){car}}
\put(10,32){\vector(-2,1){10}}
\multiput(1,0)(1,0){49}{\line(0,1){2.5}}
\multiput(5,0)(10,0){5}{\line(0,1){3.5}}
\thicklines
\put(0,0){\line(1,0){50}}
\multiput(0,0)(10,0){6}{\line(0,1){5}}
\end{picture}
\end{center}
\end{figure}

The precise code that created this figure is more advanced, and you can skip it for
now.  But note that \LaTeX\ gives you very little control of where the figure 
is actually positioned on the page.  (In the jargon, it is a `floating object'.)

\begin{chall}\label{pictureproblem} 
Explain the commands used in Figure~\ref{samplefigure}
(these occur in Lamport's book \cite{latexbook}, page 197).
\end{chall}

We can define tables in similar sort of way, such as Table~\ref{sampletable}.

\begin{table}
\caption{A test table}\label{sampletable}
\begin{center}
\begin{tabular}{ll|r|c|l}
Surname & First name & Age & Sex & Recommended treatment \\
\hline\hline
Regan & Ronald & 99 & male & Deport to Nicaragua \\
\hline
Major & John   & ? & indeterminate & Send to Coventry \\
\hline
Thatcher & Margaret & 134 & female & Banish to Los Malvinas \\
\hline\hline
\end{tabular}
\end{center}
\end{table}

\section{The table of contents}

The command \verb|\tableofcontents| is used to create a table
of contents.  It is usually placed just after the 
\verb|\maketitle| command but before the first \verb|\section|,
\verb|\chapter| or \verb|\part|.  The command does two
things.  Firstly, it instructs \LaTeX\ to create a \texttt{.toc}
file (with the same same as the main file), and as \LaTeX\ processes
your file, it stores all the data concerning what parts, chapters
and sections your document contains and on what pages they appear.
Secondly, it reads any \texttt{.toc} file that is already present
and uses the data there to create a table of contents.  Thus
the data for the table of contents is always created in the previous
run of \LaTeX, not the current one.  It follows that you must
run \LaTeX\ twice if for any reason you suspect that the
data in the \texttt{.toc} file is out of date.  (The same applies
to the other cross-referencing commands, but for a file called
the \texttt{.aux} file instead of the \texttt{.toc} file.)

\section{The label, ref, and pageref commands}

The \verb|\label|, \verb|\ref|, and \verb|\pageref| commands
are by far the most important ones.  The command \verb|\label|
sets up a label for the most `important' piece of data according to
\LaTeX\ at the given moment.  (It may be a section number, or a theorem
number, or an equation number.)  This data, and its page reference
is written to the \texttt{.aux} file.  On the second pass, the command
\verb|\ref| will be replaced by the appropriate text or number,
and the command \verb|\pageref| gives the appropriate page number.

Some things to note:
\begin{enumerate}
\item The \texttt{.aux} file might be out of date; it's worth
running \LaTeX\ two (or sometimes three) times to check.  
\item Normal references only take two passes.  Page 
references may take three.
\item Words like `equation' or `theorem' are not provided by
the \verb|\ref| command.  You have to type these yourself.\label{typeyourself}
\item Most letters or punctuation symbols are OK in labels.
\end{enumerate}

As an example of what's needed when making a cross-reference 
(see item \ref{typeyourself} above), consider the
cross-reference to page \pageref{selfref}\label{selfref},
referring to Challenge \ref{pictureproblem} on 
page \pageref{pictureproblem}.

The `tie' symbol \verb|~|, by the way, behaves like a normal space
character `\verb| |', except that it is never broken across
a line, so if you type \begin{quote}\verb|Table~\ref{sampletable}|\end{quote} 
you will not have the unfortunate effect of splitting `Table'
from the table number.  See the example of `item~\ref{typeyourself}'
in the last paragraph.

Here is the same paragraph with these tie signs instead of
spaces.

As an example of what's needed when making a cross-reference 
(see item~\ref{typeyourself} above), consider the
cross-reference on page~\pageref{selfrefa}\label{selfrefa},
referring to Challenge~\ref{pictureproblem} on 
page~\pageref{pictureproblem}.

\begin{chall}  I spent ages modifying my text so that 
`item \ref{typeyourself}' accidently went across a line-break.
How could I have used \LaTeX's more advanced commands to
achieve this effect effortlessly?
\end{chall}

\begin{thm}\label{examplethm} This text is littered
with examples of \verb|\label| and \verb|\ref|.
\end{thm}

Theorem~\ref{examplethm} is true, and to prove the point 
even more stongly,  here are some more examples.

First, an equation
\begin{equation}
e^{i\pi}=-1 \label{eipi-1}
\end{equation}

It seems that every mathematician comments at some point in their
life that Equation~\ref{eipi-1} on page~\pageref{eipi-1}
is one of the most beautiful in mathematics;  
I would hate to be an exception to the rule.

\begin{eqnarray}
1 & = & 1 \label{tri1}\\
1 +2 & = & 3 \label{tri2}\\
1 +2+3 & = & 6 \nonumber\\
1 +2+3 +4& = & 10 \label{tri4}
\end{eqnarray}

Here's a silly question that doesn't deserve an answer:
What should the number of the equation between
\ref{tri2} and \ref{tri4} be?

\section{Writing a bibliography}

Put very simply, you make a bibliography with the
\texttt{thebibliography} environment.  For example, the
bibliography for this document starts
\begin{verbatim}
\begin{thebibliography}{XXX}
\bibitem{jones} A. N. Jones, {\it Made up paper ...
...
\end{thebibliography}  
\end{verbatim}

The \verb|XXX| should be
the length of the longest printable key, which is usually
a number, and not the same as the `\LaTeX\ key' you
use to refer to it in your document.  (So for
bibliographies with between 10 and 99 entries, any two digit number
for \verb|XXX| is fine.)  Each item in the bibliography
is introduced with \verb|bibitem{latexkey}|.  \LaTeX\ automatically
assigns each item a number, which is the printed key.  To make
a reference to an item you type \verb|\cite{latexkey}|.  This creates
a number in square brackets \cite{latexbook}.

Some dos and don'ts on citations.
\begin{enumerate}  
\item \LaTeX\ has provision for you
to provide your own printable keys.  
I strongly recommend you do {\it not\/}
use them.  Keys other than numbers look ugly and are no more
helpful (sometimes less helpful).
\item Decide whether you want your references to be in citation-order
or in alphabetical order. \LaTeX\ won't do this for you.\label{biborder}
\item The point of the numbered system is that you can give references
without interrupting, breaking, or changing your text \cite{bibtexguide}.
So write as if the numbers in square brackets are not there.  Your
English should still be grammatical and understandable if they are
removed.  Don't indulge in the horrible habit of writing something like
`Jones in \cite{jones} said that this is true.'  Instead, 
write `Jones \cite{jones} said that this is true.'
\item Publishers are very fussy (with good reason) about the format
of the bibliography entries, especially punctuation and typeface.
Be careful!
\end{enumerate}

If you don't like this system, {\sc Bib}\TeX\ provides
a useful improvement, but at some extra cost of implementing
it. See my sequel to this paper~\cite{extraref}.
In particular, {\sc Bib}\TeX\ will sort the items into
alphabetical order for you \cite{extraref,bibtexguide}.

Bibliographic data is written to the \texttt{.aux} file, and so
cross-references will not be completed until the second run 
through \LaTeX.

\section{\LaTeX's warning messages}

\LaTeX\ will warn you if it thinks you need to rerun 
the program to get cross-references right.  It tends to be
over-cautious (in particular it checks all references and labels,
even ones you don't refer to by both page and number) but it's
wise not to ignore these messages.

\section{The nofiles command}\label{nofilessect}

In you really want to stop \LaTeX\ producing all those extra
files, you put the command \verb|\nofiles| in the preamble. You should do this
if the auxillary files have been created, you're sure they are right
and you don't want to destroy them.

\section{The include and includeonly commands}

These allow a long document to be split into several small parts.
This saves time when running \LaTeX, since you don't have to print out
the whole document each time you make a change, and yet \LaTeX\ will
get the page numbers and cross-referencing right.  This example
document is too short to demonstrate this feature, but you will certainly
want to use it when you are writing a book, a longish report, or a thesis.

Roughly, you put each chapter in a separate file (e.g., \texttt{chap1.tex}, and
you read this file in from the main document with the \verb|\include|
command, e.g., you type \verb|\include{chap1}|.  
The document is then \LaTeX ed
normally.  But if at a later stage you make a change to chapter 2,
and only want that chapter printed off, you put \verb|\includeonly{chap2}|
in the main document, somewhere before the `\verb|\begin{document}|'.
Then the auxillary files \texttt{chap1.aux} etc., will be read to determine
on what page chapter 2 starts, and to get all cross references right,
but only the required chapter will go into the \texttt{.dvi} file.

See Lamport's book \cite{latexbook} for details.

\section{Tips on cross-referencing}\label{sec:tips}

I find it helpful to indicate in the label what is being referred
to , for example, this section is labelled with \verb|\label{sec:tips}|
to remind me that it is a section.  This avoids the possible
pitfall of erroneously referring to Theorem~\ref{doesnotconverge}
when it is actually a `challenge'.

\section{You need to run \LaTeX\ several times}

To get the cross-referencing information correct, you
need to run \LaTeX\ several times.  Usually two or three
runs suffice, but things can vary. Some published examples 
are terrible, and I can think of one where the table
of contents is incorrect because the final run of \LaTeX\
was omitted~\cite{smullyan}. Don't make your documents 
look like this!

\begin{chall}\label{doesnotconverge}  Write a
\LaTeX\ document so that the data in the \texttt{.toc}
or the \texttt{.aux} file always changes every time you run \LaTeX.
In other words, fool \LaTeX\ so that its cross-referencing
system is always wrong. Try and make your document as short as possible.
\end{chall}

Although it is possible to fool \LaTeX, it is very difficult to
do so and almost impossible unless you deliberately try to do so.

\begin{thebibliography}{9}% we have fewer than 10 entries, so use "9"

\bibitem{jones} A. N. Jones, {\it Made up paper for the purposes of giving
an example}.  Journal of imaginary statements and  transitory phenomena,
pp. 99--98.

\bibitem{extraref} Richard Kaye, {\it Other cross-referencing tools for \LaTeX}.
Available on the web at \texttt{http://web.mat.bham.ac.uk/R.W.Kaye/latex/extraref.tex}.

\bibitem{indexntn} Richard Kaye, {\it Making an index of notation}.
Available on the web at \texttt{http://web.mat.bham.ac.uk/R.W.Kaye/latex/indexntn.tex}.

\bibitem{latexbook} Leslie Lamport, {\it \LaTeX, a document preparation 
system}.

\bibitem{bibtexguide} O. Patasnik, {\it Bib\TeX{}ing.}  Supplied with
{\sc Bib}\TeX.

\bibitem{smullyan} R. M. Smullyan, {\it Recursion Theory for Metamathematics}.
Oxford University Press, 1993.

\end{thebibliography}

\end{document}


% \documentclass[a4paper]{article}
\usepackage{makeidx}

\newcommand{\bs}{$\backslash$}
\newcommand{\fn}[1]{{\tt #1}}
\newcommand{\cn}[1]{{\tt \char"5C #1}}
\newcommand{\BibTeX}{{\sc Bib}\TeX}
\newtheorem{chall}{Challenge}[section] % chall is numbered within sections
\newtheorem{thm}[chall]{Theorem}       % thm   is numbered as for challenge
\newtheorem{quest}{Question}           % quest is numbered separately

\title{Other cross-referencing tools for \LaTeX}
\author{Richard Kaye\\School of Mathematics and Statistics\\ 
The University of Birmingham\\Birmingham B15 2TT\\U.K.}% If you
% change the document in any way, put your own name here instead of mine
\date{12th August 1998}

\makeindex

\begin{document}

\bibliographystyle{plain} % this is the recommended style.  Other possible 
                          % ones are: unsrt (same, but entries are sorted by 
                          % citation order) and abbrv (alphabetical order,
                          % but journal names etc are abbreviated).
                          % I don't recommend `alpha'.

\maketitle
\tableofcontents

\section{Introduction}

This gives examples for using supplementary \TeX\ programs, \BibTeX\ 
a bibliography database tool, and \fn{makeindex}, an index generator.

\section{\BibTeX}

\subsection{Why use \BibTeX?}

There are several reasons.\index{bibtex@\BibTeX!advantages}

\begin{itemize}
\item Publishers\index{publishers} are very fussy about how references and
      bibliographies are formatted, usually with good reason.\index{reason}
      If you use \BibTeX\ correctly, you can make any required
      stylistic changes automatically, and you just have to concentrate
      on the content of your piece.
\item \BibTeX\ automatically sorts the  references according to
      whatever scheme you require.
\item You may build a database of the papers you refer to often,
      for use in your next paper or book.
\end{itemize}
In the long run, \BibTeX\ saves time, but there are 
 disadvantages.\index{bibtex@\BibTeX!disadvantages}
\begin{itemize}
\item It takes more effort and requires more care to write the
      input file for \BibTeX.
\item Cross-references are not collated by \LaTeX\ until at least
      the third pass on your main file.
\end{itemize}

\subsection{How to use \BibTeX}

The general scheme is as follows:  you create a \LaTeX\ file as before,
with the \cn{cite} commands, but leaving off the bibliographic data.
You create one or more \BibTeX\ files (which have extension \fn{.bib})
and instead of the 
\begin{quote}
\verb|\begin{thebibliography}...\end{thebibliography}|
\end{quote}
you use a
\begin{quote}
\verb|\bibliography{file1,file2}|
\end{quote}
command, which will instruct \BibTeX\ to read files \fn{file1.bib}
and \fn{file2.bib} and put the bibliography at that point.
You will also need a\index{bibliographystyle command}
\begin{quote}
\verb|\bibliographystyle{plain}|
\end{quote}
command just after the \verb|\begin{document}| command.

The data is collected in four stages as follows.
\begin{itemize}
\item \LaTeX\ reads your main file and makes a note of all
citations that are required.  (This data goes in the \fn{.aux} 
file.)\index{aux file}
\item \BibTeX\ reads the \fn{.aux} file and then reads any required
\fn{.bib} files,\index{bib file} 
and creates a \fn{.bbl}\index{bbl file} file containing
all the references required in a format that \LaTeX\ can read.
\item \LaTeX\ reads the \fn{.bbl} file on the second run, when it
encounters the command `\cn{bibliography}'.  Note that this is usually
at the end of the document.  The \fn{.bbl} file also contains
the printable keys, and these are written to the \fn{.aux} file
at this stage.
\item \LaTeX\ is run again, and on this time will it be able
to supply the appropriate keys in the final printed text.
\end{itemize}

Thus for a \LaTeX\ document \fn{foo.tex} using \BibTeX, 
the following is the minimum
you will need to do to resolve all cross-references.
\begin{quote}
\begin{tabular}{l}
\tt latex foo \\
\tt bibtex foo \\
\tt latex foo \\
\tt latex foo
\end{tabular}
\end{quote}

In particularly bad situations,\index{bad situations} 
a further run of \LaTeX\ is required.
I have even come across \fn{.bib} files which cross-reference other
citations, and here a further {\tt bibtex foo}, {\tt latex foo},
{\tt latex foo} were required.

\subsection{Examples of the use of \BibTeX}

The \fn{.bib} file is set out in a different format to normal
\LaTeX\ files.  An example is given in \fn{xampl.bib}.  This example
is designed to show you almost all of \BibTeX's features.\index{poser}  
In practice,
you won't use them all, so here's a quick summary or two entries
in \fn{xampl.bib}

The first is the spurious paper by Aamport\index{Aamport, L. A.} 
on a parallel
document preparation system \cite{article-full}.
\begin{verbatim}
@ARTICLE{article-full,
   author = {L[eslie] A. Aamport},
   title = {The Gnats and Gnus Document Preparation System},
   journal = {\mbox{G-Animal's} Journal},
   year = 1986,
   volume = 41,
   number = 7,
   pages = "73+",
   month = jul,
   note = "This is a full ARTICLE entry",
}
\end{verbatim}
Here are some points to note.
\begin{itemize}
\item Names are usually written in full, `Leslie A. Aamport'.
Writing them `Aamport, Leslie A.' is an alternative.  The square
brackets are not really needed.  They were used here to indicate 
that the author's name {\it really is} Leslie, but this doesn't
appear in the article.  I don't recommend square brackets.
Many \BibTeX\ style will abbreviate the first given name to 
an initial anyway, and this is what is usually required.
\item  The title of the article is written fully capitalized.
Most \BibTeX\ styles will remove the capitals for articles (except for the
very first word), but not for books.  This serves as a useful way of 
distinguishing the two.  If a word in the article's title really should 
be a capital, you put it in braces, so an article title might appear
in a \fn{.bib} file as\index{brain size}
\begin{verbatim}
title={The Size of {E}instein's Brain}
\end{verbatim}
and the E will always remain a capital letter.
\item Double quotes \verb|"..."| and braces \verb|{...}| are
both possible ways to delimit fields. It doesn't matter what you
use.  The commas between field are important though.
\item Some entries do not require braces or double quotes.  These
are entries which are numbers, or are a predefined string, such as
`jul' here.
\item Page numbers are usually wriiten as \verb|pages={23--34}|
(but you can use \verb|-| instead of \verb|--|, \BibTeX\ automatically
changes \verb|-| to \verb|--| here).
\item Some of the information is not required, other parts are necessary.
You will get an error message from \BibTeX\ if you miss out 
something important.
\item `\cn{mbox}' is the correct \LaTeX\ way of ensuring a word is not 
hyphenated  and split across two lines.
\end{itemize}

Here's a `book' entry \cite{book-full}.\index{book entry}
\begin{verbatim}
@BOOK{book-full,
   author = "Donald E. Knuth",
   title = "Seminumerical Algorithms",
   volume = 2,
   series = "The Art of Computer Programming",
   publisher = "Addison-Wesley",
   address = "Reading, Massachusetts",
   edition = "Second",
   month = "10~" # jan,
   year = "{\noopsort{1973c}}1981",
   note = "This is a full BOOK entry",
}
\end{verbatim}
Note \verb|#|, used to concatenate two strings together,
and \verb|\noopsort| to fool \BibTeX\ into thinking
that the book was actually published in 1973, so that it is
ordered as such when the refenences are put into order.
(But both these features are fairly `advanced'.)

The other type of entry I use a lot is `incollection'.  Here 
is\index{incollection}
an example \cite{incollection-full}.
\begin{verbatim}
@INCOLLECTION{incollection-full,
   author = "Daniel D. Lincoll",
   title = "Semigroups of Recurrences",
   editor = "David J. Lipcoll and D. H. Lawrie and A. H. Sameh",
   booktitle = "High Speed Computer and Algorithm Organization",
   number = 23,
   series = "Fast Computers",
   chapter = 3,
   type = "Part",
   pages = "179--183",
   publisher = "Academic Press",
   address = "New York",
   edition = "Third",
   month = sep,
   year = 1977,
   note = "This is a full INCOLLECTION entry",
}
\end{verbatim}
Note the way you specify multiple authors, with an `and' between each.
\BibTeX\ will make this into something more suitable.  (What it chooses
depends on the style you are using.)

\section{{\it Makeindex}, an index processor for \LaTeX}

\subsection{When and why to use {\it Makeindex}}

Making good indices is a real pain in the neck.  There is a lot of
work collecting data, sorting it, and formatting it.  Each of these
tasks involves a lot of repetious work, but each also requires human
judgement too.  Use of a computer can reduce some of the labour,
but unfortunately not a lot.

Do not start to think about making an index until your document
is complete or incredibly nearly so. 

\subsection{How to use {\it Makeindex}}

\begin{enumerate}
\item Put the command `\verb|\usepackage{makeidx}|' in the preamble
of your document, after the \cn{documentclass} command, 
before \verb|\begin{document}|.\index{makeindex}
\item Put `\cn{makeindex}' between the \verb|\usepackage{makeidx}| command
and the command \verb|\begin{document}|.
\item Put `\cn{printindex}' where you want the index to go---usually just
before the \verb|\end{document}|.
\end{enumerate}
A \fn{.idx} file will be created, with all the index entries.
This file will contain the index item and the page number
for every \cn{index} command in the document.  For example,
if your document says \verb|\index{number theory}| and this
occurs in page~27, then the \fn{.idx} file will contain this
information and an entry will be made in the index.

The \fn{.idx} file is not read directly, but sorted and collated
to produce a \fn{.ind} file which actually produces the index.
To get this file the process is
\begin{quote}\tt
latex foo\\
makeind[e]x foo\\
latex foo
\end{quote}
Note, on some computers (such as MS-DOS machines) {\it Makeindex}
is actually called \fn{makeindx} because these machines limit commands
to eight letters.  On unix you type the full word.

This document has various \cn{index} commands scattered around it.
There are some variant of the commands, for example\index{makeindex!useof}
\begin{quote}
\verb|\index{makeindex!useof}|
\end{quote}
creates a sub-entry `use of' for the entry `makeindex' (you can have
subsubentires too);\index{makeindx|see{makeindex}}
\begin{quote}
\verb+\index{makeindx|see{makeindex}}+
\end{quote}
creates a cross-reference; and\index{alpha@$\alpha$} in
\begin{quote}
\verb+\index{alpha@$\alpha$}+
\end{quote}
The index entry is given as $\alpha$ but placed in the
appropriate order as if it had been spelt `alpha'.

See the documentation for {\it Makeindex} for more details.

\subsection{How to use {\it Makeindex} to create a glossary}

The commands \cn{makeglossary} and \cn{glossary} behave rather like
the commands \cn{makeindex} and \cn{index}, but work
with a \fn{.glo} file instead.  You can use the program
{\it Makeindex} to sort these entries if you like, but formatting
is usually more difficult and dependent on the type of 
application you have.  I'm not going to provide clues here,
but if you are interested, please ask.  I did once use
the \LaTeX\ `glossary' commands to produce an index of notation
for a book once, and this was pretty successful.  (In this
particular case I chose not to use {\it Makeindex} to sort
the entries though.\index{zzzsleep})

\bibliography{xampl}

\printindex

\end{document}


% %                          Further notes on using LaTeX2e.
%
% Supplement to "Essential LaTeX" and "Essential mathematical LaTeX"
%                                  - Richard Kaye,  Nov/Dec 1994
%
\documentclass[11pt,a4paper]{article}
\usepackage{amssymb}

\newcommand{\fn}[1]{{\tt #1}}
\newcommand{\cn}[1]{{\tt \char"5C #1}}

\title{More essential \LaTeXe}
\author{Richard Kaye\thanks{School of Mathematics and Statistics, 
The University of Birmingham, Birmingham B15 2TT, U.K.}}% If you
% change the document in any way, put your own name here instead of mine
\date{12th August 1998}

\begin{document}

\maketitle
\tableofcontents

\section{Introduction}

This document is a supplement to {\it Essential \LaTeXe} and to
{\it Essential Mathematical \LaTeX}.  The main points covered are:
\begin{itemize}
\item an overview of how the components of the \TeX~system work together;
\item the rationale behind \LaTeXe, and in what way it differs from the
previous version, \LaTeX~2.09;
\item more information on the use of fonts in \LaTeXe;
\item some ideas and pointers more advanced use of \LaTeXe, such as the 
use of user-defined commands.
\end{itemize}

As always, you are encouraged to modify this file to experiment with the
ideas here or as your own reference to suit your own use later.

\section{The components of \TeX}

Much of this is just background information relating to
how \TeX\ and its various components fit together, and the need
for a new version of \LaTeX.

\TeX~is a typesetting program which takes your document (which is a
standard text file) and typesets it according to certain built-in rules
and other rules which are automatically read in from other files.
Basically, \TeX\ is a glorified typewriter.  When it sees the letter
`a', it just goes and typesets a letter a in the current font.  

To do this it reads information about all the fonts you will use
from special files (called `\TeX~font metric' files, having 
suffix \fn{.tfm}) which tell \TeX~how large each character is
and how much space is normally left between characters.  Using this
information, \TeX~decides what characters should be put where on the 
pages, and stores this information in the output (\fn{.dvi}) file.

To further control how the output looks, \TeX~has various 
other primitive commands which can be used. You will almost never
have to work with these (although the details can be found in 
Knuth's {\it The \TeX book}).  Fortunately \TeX\ also has a mechanism to
define other commands from old ones (using a device called {\it 
macro expansion\/}), and you normally use \TeX~by first loading
a package of new commands (or macros) and working with those commands 
instead.

Packages for \TeX\ are available for typesetting almost any language,
and there are many other packages too, even one to typeset music, called
Music\TeX.

The popularity of \TeX\ is due in part to the popularity of these
packages. The original macro package for \TeX\ is (oddly) called
{\it plain \TeX\/}, even though it is much more powerful than the
primitive set of commands that are used.  Then, the American Mathematical 
Society developed another one, AMS-\TeX, and Lesie Lamport developed
\LaTeX. Arising from experience of these two, AMS-\LaTeX~was developed.
No two of these systems are compatible with each other, although
it is almost true that plain \TeX\ is a subset of the other three.
In fact, it is even worse than this, and many dialects of \LaTeX\
are in use, some with and some without the `new font selection scheme',
and even ones with this scheme are set up differently.
\LaTeX\ itself has the ability to load new packages (so-called 
style or \fn{.sty} files), and hundreds are available for all sorts of
applications.  Unfortunately standardization in these packages seems
to be rare, and the situation is rather a mess.  The latest version 
of \LaTeX, \LaTeXe, was developed to sort out the muddle with the different
dialects of \LaTeX\ and its various packages, and is discussed more below.

There are two further components of the \TeX~system that are worth 
mentioning.  The first (and most important to the user) is the family
of so-called {\it drivers\/} which convert the \fn{.dvi} file to a
printed output, or at least to something you can read.  By necessity,
these drivers are machine dependent, and you will need a different driver
for each type of printer or screen.  On the sun
network you will use \fn{dvips}, which converts the \fn{.dvi} file
to postscript and prints it normally on a laser printer, 
and \fn{xdvi} which views a \fn{.dvi}
file on an X-terminal.  Drivers are also available for most other
computers and output devices.

Roughly what a driver does is read the \fn{.dvi} file and also
information about the shape of characters in a font (usually from
\fn{.pk} files---this part is done automatically) and produces a 
graphic image which can be printed or viewed.

The other component of the \TeX~system is called {\it Metafont},
and is used to create the font files from specially written programs.
So in principle, you can design your own characters in your 
mathematical papers, but of course in practice this is time-consuming
and difficult to get to look good.  But sometimes, if \fn{dvips}
can't find a font you will find it calling Metafont automatically to 
generate it.

\TeX\ was designed for maximum portability between computers and 
different printers.  (That means it is an ideal format to send academic 
papers to a colleague by email.  But it is often wise to check that your
colleague has similar fonts and macro packages.)  A disadvantage is that
you cannot always directly use special features of your computer/printer
when working with \TeX\ and its normal fonts, for example drawing long 
diagonal lines and incorporating images is difficult to do with
\TeX's normal \fn{.tfm}, \fn{.pk} fonts.  There are several ways
to work round the problem if you can be sure you (and your colleagues)
only work with one driver (\fn{dvips} for example) or can print out
postscript files directly.

Finally, the \TeX\ software is extensive, covering typesetting in 
almost every language and in almost any application, and
just about all of this software is public domain
and/or free.  If you need some extra package it can usually be
obtained by ftp from \fn{ftp.tex.ac.uk} for example.

\section{What's new in \LaTeXe}

\LaTeX\ is not just one package, but also a rather general way of
choosing packages, loading new commands and further packages, and also
(if you should want or need to) designing document styles.  Packages
and styles are read in by \LaTeX\ as \fn{.sty} files.  You can write 
these yourself, or you can use a \fn{.sty} file written by someone
else (there are literally hundreds available).  Unfortunately this
proliferation of styles and packages caused more confusion, since
many were given the same name (or new versions of a \fn{.sty} file
were written with the same name and different functionality).  
\LaTeXe\ was devised to provide a standard base to all users.  
It was released in the summer of 1994 and is widely available, but 
unfortunately is not loaded onto every system yet!

The {\it new font selection scheme\/} (nfss) built into \LaTeXe\ makes
using special symbol fonts for mathematics, and using postscript fonts 
(rather that Knuth's `Computer Modern' fonts) much easier---provided 
of course you use a driver like \fn{dvips}. If you
want to use the additional features developed for typesetting mathematics
by the AMS, or if you want to use postscript fonts, you are strongly 
advised to use \LaTeXe.

\LaTeXe\ has a `compatibility mode', and if it sees a line like
\begin{verbatim}
\documentstyle[11pt,a4,amssymb]{article} 
\end{verbatim}
it recognises the document as a \LaTeX~2.09 document and tries
to emulate \LaTeX~2.09.  (This is usually but not always 
successful---some documents will be typeset differently, especially 
if you try to use an incompatible package.)  On the other hand a true
\LaTeXe\ document will start with
\begin{verbatim}
\documentclass[11pt,a4paper]{article}
\usepackage{amssymb}
\end{verbatim}
say, and the document can then use the new features of 
\LaTeXe\ (and typesetting should be quicker as well).

Key to \LaTeXe\ is the idea of a `document-class'---what sort of document 
it is (an article, a book, or whatever)---and the distinction between this
and a `package'.  In \LaTeX~2.09 the distinction was blurred.
There are many other enhancements made `behind the scenes' that make
writing classes, packages and styles easier and enable them to 
work more reliably and predictably.  This probably is the most important
improvement, and is part of the longer-term design of the next release
of \LaTeX, \LaTeX~3, which will not have any obvious new features
to the user, but will have a host of new features to the writer of \LaTeX\
packages.

\section{About using fonts}

The new font selection scheme (nfss) works in a significantly different
way to the old font mechanism in \LaTeX~2.09 and plain \TeX,
although in most cases the difference should be transparent to the
user.  Occasionally, a more detailed knowledge than given in 
`Essential \LaTeXe' is required.

Every font has: (a)~a family name; (b)~a series; and (c)~a shape.  
Each of these three attributes can be changed independently.  
The three main families are (1)~roman, as here, (2)~sans-serif, 
{\sffamily like this}, and (3)~typewriter, {\ttfamily like this}.  
The commands for selecting
these families are respectively \cn{rmfamily}, \cn{sffamily}
and \cn{ttfamily}.  Each font in each family has {\it series}, indicating
whether it is bold or medium-weight.
The commands for selecting
these series are  \cn{bfseries}, \cn{mdseries}.  Finally, the {\it
shape} of the font is whether the letters are upright,
italic, slanted, or caps-and-small-caps.

Here's a summary table.
{

\paragraph{\rmfamily 1. Roman family}

\rmfamily
\begin{center}\begin{tabular}{l|cccc}
& \cn{upshape} & \cn{itshape} & \cn{slshape} & \cn{scshape} \\
\hline
\cn{mdseries} %
   & \mdseries\upshape Upright % 
   & \mdseries\itshape Italic % 
   & \mdseries\slshape Slanted % 
   & \mdseries\scshape Small Caps \\
\cn{bfseries} %
   & \bfseries\upshape Upright % 
   & \bfseries\itshape Italic % 
   & \bfseries\slshape Slanted % 
   & \bfseries\scshape Small Caps \\
\hline
\end{tabular}\end{center}

\paragraph{\sffamily 2. Sans serif family}

\sffamily
\begin{center}\begin{tabular}{l|cccc}
& \cn{upshape} & \cn{itshape} & \cn{slshape} & \cn{scshape} \\
\hline
\cn{mdseries} %
   & \mdseries\upshape Upright % 
   & \mdseries\itshape Italic % 
   & \mdseries\slshape Slanted % 
   & \mdseries\scshape Small Caps \\
\cn{bfseries} %
   & \bfseries\upshape Upright % 
   & \bfseries\itshape Italic % 
   & \bfseries\slshape Slanted % 
   & \bfseries\scshape Small Caps \\
\hline
\end{tabular}\end{center}

\paragraph{\ttfamily 3. Typewriter family}

\ttfamily
\begin{center}\begin{tabular}{l|cccc}
& \cn{upshape} & \cn{itshape} & \cn{slshape} & \cn{scshape} \\
\hline
\cn{mdseries} %
   & \mdseries\upshape Upright % 
   & \mdseries\itshape Italic % 
   & \mdseries\slshape Slanted % 
   & \mdseries\scshape Small Caps \\
\cn{bfseries} %
   & \bfseries\upshape Upright % 
   & \bfseries\itshape Italic % 
   & \bfseries\slshape Slanted % 
   & \bfseries\scshape Small Caps \\
\hline
\end{tabular}\end{center}

}
\medskip

Note that is the current shape is italic, \cn{bfseries} would
select bold italic, and if text is currently upright,
\cn{bfseries} would select upright bold.  Note also that some
family/series/shape combinations are not always available.
This shouldn't create problems as suitable substitutions are made
(and a warning message issued), but if you really want these
fonts, it is reasonably straightforward to get Metafont to make them
for you.

These commands are typically used in contexts delimited with braces
in the usual way.  If you only want a \textit{small} amount of text
to be changed, you can use the commands:
\begin{itemize}
\item \verb|\textrm{Roman family}| \textrm{Roman family};
\item \verb|\textsf{Sans serif family}| \textsf{Sans serif family};
\item \verb|\texttt{Typewriter family}| \texttt{Typewriter family};
\item \verb|\textmd{Medium series}| \textmd{Medium series};
\item \verb|\textbf{Bold series}| \textbf{Bold series};
\item \verb|\textup{Upright shape}| \textup{Upright shape};
\item \verb|\textit{Italic shape}| \textit{Italic shape};
\item \verb|\textsl{Slanted shape}| \textsl{Slanted shape};
\item \verb|\textsc{Small Caps shape}| \textsc{Small Caps shape};
\item \verb|\emph{Emphasized}| \emph{Emphasized};
\end{itemize}
the last of these chooses its shape (`up' or `it') depending on context.
All of these commands will insert the appropriate extra space
if a slanted shape butts into an upright one.

To change fonts in maths mode you use the analogous commands
\begin{itemize}
\item \verb|\mathrm{Roman}|,
\item \verb|\mathnormal{Normal}|,
\item \verb|\mathcal{CALIGRAPHIC}|,
\item \verb|\mathbf{Bold}|,
\item \verb|\mathsf{Sans serif}|,
\item \verb|\mathtt{Typewriter}|,
\item \verb|\mathit{Italic}|.
\end{itemize}

\cn{mathbb} gives `blackboard bold' when the appropriate AMS
symbols package loaded.  The AMS packages also provide
cyrillic, gothic and script letters.

Here's a test: 
\[\begin{array}{ccc}
\mathrm{Roman} & \mathnormal{Normal} & \mathcal{CALIGRAPHIC} \\
\mathbf{Bold} & \mathsf{Sans\ serif} & \mathtt{Typewriter} \\
\mathit{Italic} & \mathbb{N,Z,R,C} & \mathfrak{A,B,C,D}
\end{array}\]

\cn{mathnormal} is the usual font for algebraic equations, etc.,
and the spacing between letters is designed for this.  If you want a
word typed in italic, as in \( T_{\mathit{max}} \) use \cn{mathit}.
(Compare this with \verb|T_{max}|, \( T_{max} \) which looks pretty
horrible.

There's a subtle but important point about changing fonts, often 
missed by beginners (and also people who are not beginners) at
\LaTeX, which is the so called \textit{italic correction}.
Typesetters should add a little extra space after a 
\begin{quote}
{\sl slanted\/} letter
\end{quote} 
if the next letter is upright.  If this is not done
the text looks too cramped, as in
\begin{quote}
{\sl slanted} letter.  
\end{quote}
The commands \verb|\textit|, \verb|\textsl|, and \verb|\emph| do 
this automatically, but you have to add the correction by hand with 
the command \verb|\/| when you use any of the other font-changing 
commands, including \verb|\it| and \verb|\sl|.

One final word on fonts: the commands \verb|\textrm|, \verb|\rm|
and so on are defined by the \textit{document class} in terms
of the more basic commands discussed here, so they might not work
in the same way if you change from the \verb|article| to a publisher's
special document class.  So you should probably use them rather
than the basic commands, since that way major stylistic changes can be
made with a single change.

\section{User-defined commands in \LaTeX}

\newcommand{\wake}{baba\-badalgharagh\-takam\-min\-%
arron\-nkonn\-bronn\-tonn\-er\-ronn\-tuonn\-thunn\-%
tro\-varr\-houn\-awnskawn\-too\-hoo\-hoor\-denen\-thur\-nuk}

\begin{sloppypar}
\LaTeX\ always included the factility to add new commands.
For example, you could say
\begin{verbatim} \newcommand{\wake}{baba\-badalgharagh\-takam\-min\-%
arron\-nkonn\-bronn\-tonn\-er\-ronn\-tuonn\-thunn\-%
tro\-varr\-houn\-awnskawn\-too\-hoo\-hoor\-denen\-thur\-nuk}
\end{verbatim}
in a document that makes a lot of use of the hundred-letter word 
on the first page of Joyce's \textit{Finnegans Wake}, and then you
can save yourself a lot of typing (as well as remembering where
all the suitable places to hyphenate it are) by just saying \verb|\wake|
every time you need this word as in
\begin{quote}
\verb|The fall (\wake!) of a once wallstrait oldparr\ldots|
\end{quote}
`The fall (\wake!) of a once wallstrait oldparr\ldots'
\end{sloppypar}

Note the use of the comment character (\%) to ensure that \LaTeX\
doesn't see the end-of-line symbol and therefore doesn't break up this
beautiful word into three.  (Unfortunately, \LaTeX\ does have to hyphenate 
it somewhere though.)  Note also that new commands defined this way
still suffer from the problem that any spaces following them will 
be ignored.

\newcommand{\lis}[2]{#2_{1},\ldots,#2_{#1}}
\LaTeX\ commands may take arguments too. For example, in
\begin{verbatim}
\newcommand{\lis}[2]{#2_{1},\ldots,#2_{#1}}
\end{verbatim}
we define a command \verb|\lis| of two arguments (which will replace
the \verb|#1| and \verb|#2| in the definition.  Thus
\verb|consider \( \mathbf{x} = (\lis{n}{x}) \)| becomes
`consider \( \mathbf{x} = (\lis{n}{x}) \).'

\LaTeXe\ improves the \cn{newcommand} command of \LaTeX~2.09 
by allowing you to define commands with an optional argument as well.

\newcommand{\seq}[2][n]{#2_{1},\ldots,#2_{#1}}
For example, in
\begin{verbatim}
\newcommand{\seq}[2][n]{#2_{1},\ldots,#2_{#1}}
\end{verbatim}
we define \cn{seq} with two arguments, the first being optional
(default value `n').  This means
\begin{verbatim}
For some \(i \in \{ \seq{t} \} \) we have 
\( f(i)=\seq[m]{\alpha} \)
\end{verbatim}
gives
`For some \(i \in \{ \seq{t} \} \) we have \( f(i)=\seq[m]{\alpha} \)'.

\newenvironment{qsi}[1]%
{#1 wrote,\begin{quote}\begin{sloppypar}\it}%
{\end{sloppypar}\end{quote}}
\LaTeX\ also allows you to define environments, usually
variations of existing ones.  So the definition
\begin{verbatim}
\newenvironment{qsi}[1]%
{#1 wrote,\begin{quote}\begin{sloppypar}\it}%
{\end{sloppypar}\end{quote}}
\end{verbatim}
Makes the following \LaTeX\ source
\begin{verbatim}
\begin{qsi}{Joyce}
The fall (\wake!) of a once wallstrait oldparr\ldots
\end{qsi}
\end{verbatim}
yield: \begin{qsi}{Joyce}
The fall (\wake!) of a once wallstrait oldparr\ldots
\end{qsi}

\section{Packages}  

The best feature of \LaTeXe\ is its ability to load
and use packages from various sources in a smooth, uniform way.  
Several are bundled with the
new \LaTeX, including ones to print graphics, and one to vary 
`theorem' environments. 
The AMS packages now work well with \LaTeX\ and are highly recommended
for tricky mathematical typesetting.  For example, if you require
the AMS fonts, you are strongly advised to use \LaTeXe\ with the
appropriate package.

For further information on \LaTeXe\ and its packages, see
`\LaTeXe\ for authors' and `AMS-\LaTeX\ Version 1.2 User's guide'
which are on fourier and copies can be made available.

\end{document}


\section{Errors}

When you create a new input file for \LaTeX{} you will probably make mistakes.
Everybody does, and it's nothing to be worried about.  As with most computer
programs, there are two sorts of mistake that you can make: those that \LaTeX{}
notices and those that it doesn't.  To take a rather silly example, since
\LaTeX{} doesn't understand what you are saying it isn't going to be worried if
you mis-spell some of the words in your text.  You will just have to accurately
proof-read your printed output.  On the other hand, if you mis-spell one of
the environment names in your file then \LaTeX{} won't know what you want it
to do.

When this sort of thing happens, \LaTeX{} prints an error message on your
terminal screen and then stops and waits for you to take some action.
Unfortunately, the error messages that it produces are rather user-unfriendly
and not a little frightening.  Nevertheless, if you know where to look they
will probably tell you where the error is and went wrong.

Consider what would happen if you mistyped \verb|\begin{itemize}| so that it
became \verb|\begin{itemie}|.  When \LaTeX{} processes this instruction, it
displays the following on your terminal:
\begin{quote}\footnotesize\begin{verbatim}
LaTeX error.  See LaTeX manual for explanation.
              Type  H <return>  for immediate help.
! Environment itemie undefined.
\@latexerr ...for immediate help.}\errmessage {#1}
                                                  \endgroup
l.140 \begin{itemie}

?
\end{verbatim}\end{quote}
After typing the `?' \LaTeX{} stops and waits for you to tell it what to do.

The first two lines of the message just tell you that the error was detected by
\LaTeX{}. The third line, the one that starts `!' is the {\em error indicator}.
 It
tells you what the problem is, though until you have had some experience of
\LaTeX{} this may not mean a lot to you.  In this case it is just telling you
that it doesn't recognise an environment called \fn{itemie}.
The next two lines tell you what
\LaTeX{} was doing when it found the error, they are irrelevant at the moment
and can be ignored. The final line is called the {\em error locator}, and is
a copy of the line from your file that caused the problem.
It start with a line number to help you to find it in your file, and
if the error was in the middle of a line it will be shown
broken at the point where \LaTeX{} realised that there was an error.  \LaTeX{}
can sometimes pass the point where the real error is before discovering that
something is wrong, but it doesn't usually get very far.

At this point you could do several things.  If you knew enough about \LaTeX{}
you might be able to fix the problem, or you could type `X' and press the
return key to stop \LaTeX{} running while you go and correct the error.  The
best thing to do, however, is just to press the return key.  This will allow
\LaTeX{} to go on running as if nothing had happened.  If you have made one
mistake, then you have probably made several and you may as well try to find
them all in one go.  It's much more efficient to do it this way than to run
\LaTeX{} over and over again fixing one error at a time. Don't worry about
remembering what the errors were---a copy of all the error messages is being
saved in a {\em log\/} file so that you can look at them afterwards.  See your
{\em local guide\/} to find out what this file is called.

If you look at the line that caused the error it's normally obvious what the
problem was.  If you can't work out what you problem is look at the hints
below, and if they don't help consult Chapter~6 of the manual.  It contains a
list of all of the error messages that you are likely to encounter together with
some hints as to what may have caused them.

Some of the most common mistakes that cause errors are
\begin{itemize}
\item A mis-spelt command or environment name.
\item Improperly matched `\verb|{|' and `\verb|}|'---remember that they should
 always
come in pairs.
\item Trying to use one of the ten special characters \verb|# $ % & _ { } ~ ^|
and \verb|\| as an ordinary printing symbol.
\item A missing \verb|\end| command.
\item A missing command argument (that's the bit enclosed in '\verb|{|' and
`\verb|}|').
\end{itemize}

One error can get \LaTeX{} so confused that it reports a series of spurious
errors as a result.  If you have an error that you understand, followed by a
series that you don't, then try correcting the first error---the rest
may vanish as if by magic.

Sometimes \LaTeX{} may write a {\tt *} and stop without an error message.  This
is normally caused by a missing \verb|\end{document}| command, but other errors
can cause it.  If this happens type \verb|\stop| and press the return key.

Finally, \LaTeX{} will sometimes print {\em warning\/} messages.  They report
problems that were not bad enough to cause \LaTeX{} to stop processing, but
nevertheless may require investigation.  The most common problems are
`overfull' and `underfull' lines of text.  A message like:
\begin{quote}\footnotesize\begin{verbatim}
Overfull \hbox (10.58649pt too wide) in paragraph at lines 172--175
[]\tenrm Mathematical for-mu-las may be dis-played. A dis-played
\end{verbatim}\end{quote}
indicates that \LaTeX{} could not find a good place to break a line when laying
out a paragraph.  As a result, it was forced to let the line stick out into the
right-hand margin, in this case by 10.6 points.  Since a point is about 1/72nd
of an inch this may be rather hard to see, but it will be there none the less.

This particular problem happens because \LaTeX{} is rather fussy about line
breaking, and it would rather generate a line that is too long than generate a
paragraph that doesn't meet its high standards.  The simplest way around the
problem is to enclose the entire offending paragraph between
\verb|\begin{sloppypar}| and \verb|\end{sloppypar}| commands.  This tells
\LaTeX{} that you are happy for it to break its own rules while it is working on
that particular bit of text.

Alternatively, messages about ``Underfull \verb|\hbox|'es'' may appear.
These are lines that had to have more space inserted between
words than \LaTeX{} would have liked.  In general there is not much that you
can do about these.  Your output will look fine, even if the line looks a bit
stretched.  About the only thing you could do is to re-write the offending
paragraph!

\section{A Final Reminder}

You now know enough \LaTeX{} to produce a wide range of documents.  But this
document has only scratched the surface of the
things that \LaTeX{} can do.  This entire document was itself produced with
\LaTeX{} (with no sticking things in or clever use of a photocopier) and even
it hasn't used all the features that it could.  From this you may get some
feeling for the power that \LaTeX{} puts at your disposal.

Please remember what was said in the introduction: if you \textit{do} have a
complex document to produce then \textbf{go and find out more 
about \LaTeX{} first}.  You will be
wasting your time if you rely only on what you have read here.

One other warning: having
dabbled with \LaTeX{} your documents will never be the same again\ldots

\end{document}
