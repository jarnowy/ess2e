\documentclass[a4paper,landscape]{slides}
\newcommand{\bs}{$\backslash$}
\newcommand{\fn}[1]{{\tt #1}}

\begin{document}

\title{Introduction to \LaTeX}
\author{Richard Kaye}
\date{Friday October 18th, 2002}
\maketitle

\newpage

\textbf{Making slides using \LaTeX}

These slides 
\begin{itemize}
\item introduce \LaTeX\ to newcomers
\item outline some of the goals and themes
for successful use of \LaTeX.  (Important 
information whatever your \LaTeX\ skills are!)
\item provides an example of how you might prepare
slides for a talk or lecture using \LaTeX
\end{itemize}

\newpage

\textbf{Typesetting} 

\TeX\ is a \textit{typesetting program}, not a word processor.

A typesetting program decides where symbols, such as 
letters, punctuation etc., should appear.  

Typesetting is a highly developed art with many subtle points, 
and is \textit{much} more difficult to get right than it looks.  

\begin{itemize}\bf
\item A document is only as good as its content.
\item The \textit{only} objective in typesetting a
technical document is to make your document as easy to read as 
possible.
\end{itemize}

\newpage

\textbf{Document design}

Even harder than simple typesetting is \textit{document design}.
Bad design is irritating and detracts from your content.
Good design is `invisible' and rarely noticed.

\newpage

\textbf{\TeX}

\TeX\ is a freely available typesetting program that 
works in some way or other on almost all computers
with almost all printers.

\textbf{Advantages} \TeX\ is:
 free software; presumed bug-free;
 standard; everyone has it (or can get it);
 allows you to typeset documents to the 
 very highest standards.

\textbf{Disadvantages} \TeX\ is: user-unfriendly;
 sometimes difficult to use well.

\newpage

\textbf{\LaTeX}

\LaTeX\ is a set of packages extending \TeX\ which 
takes care of much of the document design.

It requires you to describe the \textbf{logical structure} 
of your document (sections, subsections, titles, etc.)~and 
leave \LaTeX\ to fit these to some built-in document design.  

\begin{itemize}
\item \LaTeX\ manages cross-references (theorem numbers, 
bibliography, page references, etc.)~automatically.
\item There is a vast library of styles and packages you may use, 
or you may define your own.
\end{itemize}

\newpage

The following tasks are listed in approximate order of difficulty.
\begin{enumerate}\itemsep=0pt
\item Simple text.
\item Simple mathematics.
\item Alignment, tables.
\item More complicated mathematics, e.g., involving alignments.
\item Document design: choice of typefaces, spacing, margins.
\item Typography: design of typefaces.
\end{enumerate}

\newpage

\textbf{What do I need for this course?}

You need access to one of the department's solaris servers.\\
All materials for the course are on-line at\\
\texttt{http://web.mat.bham.ac.uk/R.W.Kaye/latex/}

\begin{small}

If you want to use \LaTeX\ on your computer at home\ldots

Mik\TeX\ (\TeX\ for Win32 operating systems) is at\\
\texttt{http://www.miktex.org/}.  \TeX\ is available
for other OSs too.

You will need a text editor.  For Win32, WinEdt 
(\texttt{http://www.winedt.com/}) can
be recommended, but is shareware (i.e., you have to pay for
it---\$30 for students). If you prefer not to
pay, try PFE 
(\texttt{http://www.lancs.ac.uk/people/steveb/cpaap/pfe/default.htm})
or emacs (\texttt{http://www.gnu.org/software/emacs/emacs.html}).

\end{small}

\newpage

\textbf{What are we trying to do?}

Your \LaTeX\ documents should describe the
structure of the document and content \textit{not}
its visual appearance.

E.g., tell \LaTeX\ where sections, subsections, etc., start
and end and resist the temptation to tell it where
line breaks and pages breaks should be.

Here are a few golden rules:
\begin{itemize}\itemsep=0pt
\item whatever you do, be consistent
\item keep things simple
\item use the LaTeX environments and commands given wherever possible
      and nest environments in the most logical way
\item use \LaTeX\ commands and environments to describe structure
\item whenever you find you are using the same construction or notation 
regularly, write a macro or environment for it
\item avoid visual formatting
\item don't use plain \TeX\ commands
\item if you do need to change the default styles or visually format 
your document differently, write a special custom class or style file
\end{itemize}

This principle of presenting logical structure rather than visual
formatting is the same as for (good) HTML, XHTML, XML.

\newpage

\textbf{Making \LaTeX\ slides}

To make slides, you use the \verb|slides| document class.

Most of the usual text formatting commands are available, 
as you will see, but note that sectioning commands (\verb|\paragraph|,
\verb|\section|, etc.)~are not available.  You have to do these
`by hand' with \verb|\textbf|, etc.  You may find you make quite
a lot more use of \verb|\newpage| and \verb|\\| than normal.

\newpage

You will note that the fonts are different too.

Many people think that the slides package uses 
the wrong font or wrong size.  

Be very careful about changing these defaults.  
Smaller fonts become almost impossible to read
from the back of a lecture theatre.  (If you must, 
try it out.  But by all means \textit{increase}
the size of the fonts though!)  

Although it is not the usual font for textbooks,
the sans-serif font used here is less cluttered
and easier to read at a distance.

\end{document}
