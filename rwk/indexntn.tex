\documentclass[a4paper]{report}

\usepackage{makeidx}

\newcommand\Aut{\mathop{\mathrm{Aut}}\nolimits}

% index of notation
\def\glossaryentry#1#2{\item #1, #2}
\makeatletter
\def\printnotation{{%
\def\indexname{Index of notation}
\begin{theindex}
\@input{\jobname.ntn}
\end{theindex}
}}
\makeatother

\makeglossary

\title{Making an index of notation}
\author{Richard Kaye}
\date{12th August 1998}

\begin{document}
\maketitle
\chapter{Making an index of notation}

This document uses the \texttt{makeidx} package to generate an index
of notation, ordered in the order of appearance in the book or report.
(This is the order I would recommend anyway!)  

To generate the finished \texttt{dvi} file do
\begin{verbatim}
latex testntn
cp testntn.glo testntn.ntn
latex testntn
\end{verbatim}
(The UNIX \texttt{cp} command simply copies one file onto another.  Under
other systems use some alternative command.)

That's all there is to it!

If you really want to change the order of entries in your index of 
notation, you'll have to do that somehow yourself. (Replace the 
\texttt{cp} command with something else that
generates a \texttt{.ntn} file from a \texttt{.glo} file.)  You can of
course do this by hand in an editor and then do `save as'.

This is a test: $M_{22}.$\glossary{$M_{22},M_{23},M_{24},\Aut M_{22}$}
Testing testing 123\ldots More testing.\glossary{123}

\printnotation

\end{document}

