\documentclass[a4paper]{article}
\usepackage{amsthm}

\title{Using the \texttt{amsthm} package}
\author{Richard Kaye}
\date{12th August 1998}

\newcommand\cn[1]{$\mathtt\backslash$\texttt{#1}}
\newtheorem{prop}{Proposition}
\newtheorem{thm}{Theorem}[section]
\newtheorem{lem}[thm]{Lemma}
\newtheorem*{Zorn}{Zorn's Lemma}
\theoremstyle{definition}
\newtheorem{dfn}{Definition}
\theoremstyle{remark}
\newtheorem*{rmk}{Remark}
\theoremstyle{remark}
%\newtheorem*{proof}{Proof}

\begin{document}

\maketitle
\section{Introduction}

The ams\LaTeX\ \verb|amsthm| package extends \LaTeX's \verb|\newtheorem| command
(see `Cross references in \LaTeX' and the file \verb|crossref.tex|), and is
very useful.  The following is taken from the documentation.
\begin{quotation}
The \texttt{amsthm} package is loosely derived from \texttt{theorem.sty}
    version 2.1c; it adds \cn{newtheorem*} for unnumbered environments
    and changes the extensibility support for loading extra theorem
    styles from external files: now it is done through the package
    option mechanism.

Here are some examples showing the kinds of theorem environment
    declarations that are possible with the \verb|amsthm| package.
\begin{verbatim}
  \newtheorem{prop}{Proposition}
  \newtheorem{thm}{Theorem}[section]
  \newtheorem{lem}[thm]{Lemma}
  \newtheorem*{Zorn}{Zorn's Lemma}
  
  \theoremstyle{definition}
  \newtheorem{dfn}{Definition}

  \theoremstyle{remark}
  \newtheorem*{rmk}{Remark}
\end{verbatim}

The first four statements all define environments using the default
     theorem style (`plain'), since there is no prefatory
     \cn{theoremstyle} declaration. The first statement defines an
     automatically numbered \texttt{prop} environment whose headings will
     look like this: Proposition 1, Proposition 2, and so forth. The
     second statement defines an environment \texttt{thm} with numbers
     subordinate to section numbers, so the headings will look like
     this: Theorem 1.1, Theorem 1.2, Theorem 1.3, \dots, (in section 2:)
     Theorem 2.1, Theorem 2.2, and so forth. The third statement defines
     a \texttt{lem} environment whose numbers will interleave in sequence
     with the theorem numbers: Theorem 1.3, Lemma 1.4, Lemma 1.5,
     Theorem 1.6, and so forth. The fourth statement defines a special
     unnumbered lemma named `Zorn's Lemma'. The remaining two
     \cn{newtheorem} statements have no special features except for the
     \cn{theoremstyle} declarations that cause the \texttt{dfn} and
     \texttt{rmk} environments to have some differences in appearance.

There are three basic styles provided: The `plain' style produces
     bold headings and italic body text; the `definition' style produces
     bold headings and normal body text; the `remark' style produces
     italic headings and normal body text.
\end{quotation} 

The examples here give the following.

\begin{prop} This is a proposition.
\end{prop}

\begin{thm} This is a theorem.
\end{thm}

\begin{lem} This is a lemma.
\end{lem}

\begin{Zorn} This is Zorn's lemma.
\end{Zorn}

\begin{dfn} This is a definition.
\end{dfn}

\begin{rmk} This is a remark.
\end{rmk}

I suggest you use these commands as in the examples given, using whatever
`theorem style' you prefer for theorems, lemmas, remarks, examples,
definitions and so on.  (You don't \textit{have} to use the theorem 
style `definition' for definitions only!)  Personally, I prefer to use 
the `definition' style and `remark' style for everything, and not use 
the `plain' style at all.

The \texttt{amsthm} package also has an environment for proofs, 
which you use as follows.
\begin{verbatim}
\begin{proof} We prove Lemma~5 by first considering the
value of $x$. \ldots
\end{proof}
\end{verbatim}
giving
\begin{proof} We prove Lemma~5 by first considering the
value of $x$. \ldots
\end{proof}

For more advanced \LaTeX ing, there are some more features added to
the \verb|amsthm| package.  You most likely will never need them, but in 
case you do, here is the documentation.

\begin{quotation}
     A \cn{swapnumbers} command allows theorem numbers to be swapped to
     the front of the theorem headings. Putting \cn{swapnumbers} in your
     document preamble will cause \emph{all following} \cn{newtheorem}
     statements to produce number-first headings. (To provide maximum
     control, \cn{swapnumbers} is designed so that it can be used more
     than once; each time it is used, theorem numbers will be swapped to
     the opposite side for all following \cn{newtheorem} statements. But
     rarely will it need to be invoked more than once per document.)

There is a \cn{newtheoremstyle} command provided to make the
     creation of custom theoremstyles fairly easy.

Usage:
\begin{verbatim}
                    #1
  \newtheoremstyle{NAME}%
      #2          #3          #4
    {ABOVESPACE}{BELOWSPACE}{BODYFONT}%
      #5      #6        #7         #8
    {INDENT}{HEADFONT}{HEADPUNCT}{HEADSPACE}%
      #9
    {CUSTOM-HEAD-SPEC}
\end{verbatim}
     Leaving the `indent' argument empty is equivalent to entering
     \verb|0pt|. The `headpunct' and `headspace' arguments are for the
     punctuation and horizontal space between the theorem head and the
     following text. There are two special values that may be used for
     `headspace': a single space means that a normal interword space
     should be used; ``\cn{newline}'' means that there should be a line
     break after the head instead of horizontal space. The
     `custom-head-spec' argument follows a special convention: it is
     interpreted as the replacement text for an internal three-argument
     function \cn{thmhead}, i.e., as if you were defining
\begin{verbatim}
  \renewcommand{\thmhead}[3]{...#1...#2...#3...}
\end{verbatim}
     but omitting the initial \verb|\renewcommand{\thmhead}[3]|. The three
     arguments that will be supplied to \cn{thmhead} are the name,
     number, and optional note components. Within the replacement text
     you can (and normally will want to) use other special functions
     \cn{thmname}, \cn{thmnumber}, and \cn{thmnote}. These will print
     their argument if and only if the corresponding argument of
     \cn{thmhead} is nonempty. For example
\begin{verbatim}
  {\thmname{#1}\thmnumber{ #2}\thmnote{ (#3)}}
\end{verbatim}
     This would cause the theorem note \#3 to be printed with a
     preceding space and enclosing parentheses, if it is present, and if
     it is absent, the space and parentheses will be omitted because
     they are inside the argument of \cn{thmnote}.

Finally, if you have an extra bit of arbitrary code that you want
     to slip in somewhere, the best place to do it is in the `body font'
     argument.

The \cn{newtheoremstyle} command is designed to provide, through a
     relatively simple interface, control over the style aspects that
     are most commonly changed. Clearly it cannot serve for all possible
     theorem styles. Therefore there is a second interface provided to
     allow arbitrary theorem styles by reading suitable definitions from
     a separate file whose name ends with \texttt{.thm}. If the desired
     style is far from any of the basic styles provided by the
     \texttt{amsthm} package, writing the definitions may require some
     expertise in \LaTeX's macro language.

Suppose that you wanted to make a theorem style `exercise' for
     exercises. Create a file called \texttt{exercise.thm} and in it define
     a command \texttt{th@exercise} following the form of the commands
     \texttt{th@plain}, \texttt{th@definition}, \texttt{th@remark} given below. Then
     to use the new style, write
\begin{verbatim}
  \usepackage[exercise]{amsthm}
  ...
  \theoremstyle{exercise}
\end{verbatim}
     Similarly, it's possible to place a group of related
     \cn{newtheoremstyle} statements in a \texttt{.thm} file, let's say
     \texttt{stygroup.thm}, so that they could be loaded on demand in
     various documents by
\begin{verbatim}
  \usepackage[stygroup]{amsthm}
\end{verbatim}

This strategy fails if you want to load a \texttt{.thm} file in a
     document preamble and the \texttt{amsthm} package has already been
     loaded in the documentclass (e.g., \texttt{amsart}). Then you need to
     use a statement such as
\begin{verbatim}
  \PassOptionsToPackage{stygroup}{amsthm}
\end{verbatim}
     \emph{before} the \cn{documentclass} command.
\end{quotation}
\end{document}
